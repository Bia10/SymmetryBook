%% this chapter sets up the foundational system, which is Univalent Foundations
\label{ch:univalent-mathematics}

\section{What is a type?}
\label{sec:what-is-a-type}

In some computer programming languages, all variables are introduced along with a declaration of the type of thing they will refer to.  For example,
one may encounter types such as $bool$, $string$, $int$, and $real$, describing Boolean values, character strings, 32 bit integers, and 64 bit
floating point numbers.  The types are used to determine which statements of the programming language are grammatically well-formed.
For example, if $s$ of type $string$ and $x$ is of type $real$, we may write $1/x$, but we may not write $1/s$.

Types occur in mathematics, too, and are used in the same way: all variables are introduced along with a declaration of the type of thing they
will refer to.  For example, one may say ``consider a point $P$ of the plane'', ``consider a line $L$ of the plane'', ``consider a hexagon
$H$'', or ``consider a graph $G$''.  The types are used to determine which mathematical statements are grammatically well-formed.  For example,
one may write ``$P$ lies on $L$'' or ``$L$ passes through $P$'', but not ``$L$ lies on $P$''.

In \emph{univalent mathematics}, types are used to classify all mathematical objects.  Every mathematical object is an element of some (unique)
type.

One expresses the statement that an ``element'' $a$ is of ``type'' $X$ by writing $a:X$.  Using that notation, each variable is introduced along
with a declaration of the type of thing it will refer to, and the declared types of the variables are used to determine which statements of the
theory are grammatically well-formed.

There are enough ways to form new types from old ones to provide everything we need to write mathematics.

If $X$ and $Y$ are types, there will be a type whose elements serve as \emph{functions} from $X$ to $Y$; the notation for it is $X \to Y$.  Thus
when we write $f : X \to Y$, we mean that $f$ is an element of the type $X \to Y$, and we are saying that $f$ is a function from $X$ to $Y$.

Functions behave as one would expect, and one can make new ones in the usual way.

To provide an example of making new functions in the usual way, consider functions $f : X \to Y$ and $g : Y \to Z$.  We define their composite
$g \circ f : X \to Z$ by setting $g \circ f \defeq (a \mapsto g(f(a)))$.  Such definitions are to be regarded as syntactically transparent in
our formal system, in the sense that two formal expressions will be regarded as being \emph{the same by definition} if they yield the same formal
expression after the definitions of all the symbols within them are completely expanded.  Given two expressions that are the same by definition,
we may replace one with the other in any other expression, at will.  Here is an example: consider functions $f : X \to Y$, $g : Y \to Z$, and $h
: Z \to W$.  Then $(h \circ g) \circ f$ and $h \circ (g \circ f)$ are the same by definition, since applying the definitions within expands both
to $a \mapsto h(g(f(a)))$.

One may define the identity function $id_X : X \to X$ by setting $id_X \defeq (a \mapsto a)$.  Application of definitions shows that $f \circ
id_X$ is the same as $a \mapsto f(a)$, which, by convention, is to be regarded as the same as $f$.  A similar computation applies to $id_Y \circ
f$.

In the following sections we will expose various other elementary types and elementary ways to make new types from old ones.

\section{The type of natural numbers}
\label{sec:natural-numbers}

Here are Peano's rules \citep{peano-principia} for constructing the natural numbers in the form that is used in type theory.
\begin{enumerate}
\item[P1:] there is a type called $\NN$ (whose elements will be called natural numbers);
\item[P2:] there is an element of $\NN$ called $0$;
\item[P3:] if $m$ is a natural number, then there is also a natural number $S(m)$, called the \emph{successor} of $m$;
\item[P4:] given a family of types $X(m)$ depending on a parameter
  $m$ of type $\NN$, in order to define a family $f(m) : X(m)$ of elements of each of them it suffices to provide an element $a$ of $X(0)$ and
  to provide, for each $m$, a function $g_m : X(m) \to X(S(m))$.  (The resulting function $f$ may be regarded as having been defined inductively
  by the two declarations $f(0) \defeq a$ and $f(S(m)) \defeq g_m(f(m))$.)
\end{enumerate}
\nopagebreak
You may recognize rule P4 as ``the principle of mathematical induction'' or as ``defining a function by recursion''.  We may also refer to it
simply as ``induction for $\NN$''.  Notice that the two cases in an inductive definition correspond to the two ways of introducing elements of
$\NN$ via the use of rules P2 and P3.

Here is an example of defining a function by recursion using induction for $\NN$.  We define the factorial function $f : \NN \to \NN$ by
defining $X(m)$ to be $\NN$ for all $m$ and by setting $f(0) \defeq 1$ and setting $f(S(m)) \defeq (m+1) \cdot f(m)$.  One can infer that the
function $g_m$ of rule P4 is $n \mapsto (m+1) \cdot n$.

We introduce the following definitions.
\begin{align*}
 1 & \defeq S(0) \\
 2 & \defeq S(1) \\
 3 & \defeq S(2) \\
 4 & \defeq S(3)
\end{align*}

We may use induction to define the sum $m+n$ of two natural numbers, as a natural number.  We handle the two possible cases for the argument $m$
as follows: we define $0+n \defeq n$, and we define $(S(m))+n \defeq S(m+n)$.  Application of definitions shows, for example, that $2+2$ and $4$
are the same by definition, because they both reduce to $S(S(S(S(0))))$.

Before we can write an equation such as $2+2=4$, we must introduce a formal treatment of equality in type theory.  We do that in the next section.

\section{Identity types}
\label{sec:identity-types}

One of the most important types is the \emph{identity type}, which implements the intuitive notion of equality; the reader may be more
comfortable if we call it the \emph{equality type}, at least initially.  Identity (or equality) between two elements may be considered only when
the two elements are of the same type; we shall have no need to compare elements of different types.

Here are the rules for constructing equality types.
\begin{enumerate}
\item[E1:]
  for any type $X$ and for any elements $a$ and $b$ of it, there is a type $a=b$;
\item[E2:] for any type $X$ and for any element $a$ of it, there is an element $\refl a$ of type $a=a$ (the name $\refl{}$ comes from the word
  ``reflexivity'')
\item[E3:] for any type $X$ and for any element $a$ of it, given a family of types $P(b,e)$ depending on parameters $b$ of type $X$ and $e$ of type
  $a=b$, in order to define elements $f(b,e) : P(b,e)$ of all of them it suffices to provide an element $p$ of $P(a,\refl a)$.  The resulting
  function $f$ may be regarded as having been completely defined by the single definition $f(a,\refl a) \defeq p$.
\end{enumerate}

We will refer to an element $i$ of $a=b$ as an {\em identification} of $a$ with $b$, because there may be more than one of them.  When we know
that there can be at most of them, we will refer to $i$ as a {\em proof} that $a$ is equal to $b$.

We see from rule E2 that $\refl{S(S(S(S(0))))}$ serves as a proof of $2+2=4$, as do $\refl 4$ and $\refl{2+2}$.  A student might wish for a
more detailed proof of that equation, but as a result of our convention above that definitions are syntactically transparent, the application of
definitions, including inductive definitions, is regarded as a trivial operation.

We will refer to rule E3 as ``induction for equality''.  It says that to prove something about (or to construct something from) every proof that
$a$ is equal to something else, it suffices to consider the special case where the proof is the trivial proof that $a$ is equal to itself, i.e.,
where the proof is $\refl a : a=a$.  Notice that the single case in such an induction corresponds to the single way of introducing elements of
equality types via rule E2, and compare that with P4, which dealt with the two ways of introducing elements of $\NN$.
%% ???
Intuitively, the induction principle for equality amounts to saying that the element $\refl a$ ``generates'' the system of types $a=b$, as $b$
ranges over elements of $A$.

We may use induction to prove symmetry of equality.  In accordance with our discussion of implication above, we show how to produce an element
of $b=a$ from an element $p$ of $a=b$, for any $b$ and $p$.  By induction (letting $P(b,e)$ be $b=a$ for any $b$ of type $X$ and for any $e$ of
type $a=b$, for use in rule E3 above), it suffices to produce an element of $a=a$; we choose $\refl a$ to achieve that.

Transitivity of equality is established the same way.  For each $a,b,c:X$ and for each $p:a=b$ and for each $q:b=c$ we want to produce an
element of type $a=c$.  By induction on $q$ we are reduced to the case where $c$ is $b$ and $q$ is $\refl b$, and we are to produce an element
of $a=b$.  The element $p$ serves the purpose.  Notice the similarity of this inductive definition with the definition given above of the sum
$m+n$.

Now we state our symmetry result a little more formally.

\begin{definition}\label{def:eq-symm}
  For any type $X$ and for any $a,b:X$, let $\symm_{a,b} : (a=b) \to (b=a)$ be the function defined by induction by setting
  $\symm_{a,a}(\refl a) \defeq \refl a$.
\end{definition}

Similarly, transitivity is formulated as an inductive definition for a function $\trans_{a,b,c} : (a=b) \to ((b=c) \to (a=c))$.  We may
abbreviate $(\trans_{a,b,c}(p))(q)$ as $p*q$.

Associativity of transitivity is formulated and established similarly.  We leave that as an exercise.

One frequent use of elements of identity types is in \emph{substitution}.  Let $X$ be a type, and let $T(x)$ be a family of types depending on a
parameter $x:X$.  Suppose $x,y:X$ and $e:x=y$.  Then there is a function of type $T(x) \to T(y)$. We define one specific such function by induction, by taking its value to be the identity function on $T(x)$ in the case of $\refl{x}:x=x$.
\begin{definition}\label{def:transport} The function
  \[ 
  \trp_{T,e} : T(x) \to T(y)
  \]
  is defined by induction setting $\trp_{T,\refl{x}} (t) \defeq t$.
\end{definition} 
The function thus defined may be called 
\emph{the transport function in the type family $T$ along the path $e$}, 
 or, less verbosely, \emph{transport}.
 We may also simplify the notation to just $\trp_e$.
The transport functions behave as expected: transport along the composition
$e\cdot e'$ is the composition of the two transport functions (to be
 proved by induction).

When the types $T(x)$ may have more than one element, 
we may regard an element of $T(x)$ as providing additional {\em structure} on $x$. 
In that case, we will refer to the transport function $T(x) \to T(y)$ as 
\emph{transport of structure} from $x$ to $y$. 

Take, for example, $T(x)\defeq x=x$. 
Then $\trp_e$ is of type $x=x \to y=y$ and transports the
symmetries of $x$ to the symmetries of $y$.

By contrast, when the types
$T(x)$ have at most one element, we may regard an element of $T(x)$ 
as providing a proof of a property of $x$. In that case, the transport
function $T(x) \to T(y)$ provides a way to establish a claim about $y$ 
from a claim about $x$, so we will refer to it as \emph{substitution}.  In
other words, elements that can be identified have the same properties.

\section{Product types}
\label{sec:product-types}
Our type theory will also contain \emph{products} of types. 
By this we mean if $X$ is a type and $Y(x)$ is a family of types indexed by a
parameter $x$ of type $X$, then there will be a type $\prod_{x:X} Y(x)$ 
whose elements $f$ are functions that provide elements $f(a)$ of type
$Y(a)$, one for each $a:X$. We may refer to $X$ as the 
\emph{index type} of the product. 
A function $f : X \to Y$ is essentially the same thing as a function $f$ 
of type $\prod_{x:X} Y$, where the product is formed using a constant family of types.

The basic way of constructing an element of a product type
is by explicit definition, as we have seen before.

If two functions $f$ and $g$ of type $\prod_{x:X} Y(x)$ are equal, 
then they have equal values, i.e., for every element $x$ of $X$, 
we may conclude that $f(x) = g(x)$.
(This can be proven by induction or by substitution, as suggested in WHICH EXERCISE.)
Conversely, a basic principle, called `function extensionality', 
asserts that if for every element $x$ of $X$, 
we know that $f(x) = g(x)$, then we may conclude that $f=g$.

\section{Inductive types}
\label{sec:inductive-types}

There are other examples of types that are conveniently presented as 
inductive definitions, in the style we have seen with the natural numbers
and the equality types.  We start by the finite types, and then
present several constructions defining new types from old ones.
For each of these constructions we explain what it means for two 
elements of the newly constructed type to be equal in terms of
equality in the constituent types.

\subsection{Finite types}
\label{sec:finite-types}
Firstly, there is the ``empty'' type, called $\emptyset$, defined inductively, with no way to construct elements provided in the inductive
definition.  The inductive principle for $\emptyset$ says that to prove something about (or to construct something from) every element of
$\emptyset$, it suffices to consider no special cases (!).  Hence, every statement about an arbitrary element of $\emptyset$ can be proven. (This is called the Ex Falso
rule in traditional logic.) As
an example, we may prove that any two elements $x$ and $y$ of $\emptyset$ are equal by using induction on $x$.

An element of $\emptyset$ will be called an \emph{absurdity}, and the negation $\neg P$ of a proposition $P$ will be implemented as the function
type $P \to \emptyset$.  This is sensible, because an element of $\neg P$ could be applied to an element of $P$ to produce an element of
$\emptyset$, i.e., an absurdity.

Another appropriate name for $\emptyset$ is $\false$.

We may also construct a function $\false \to X$, for any type $X$, by induction, showing that from an absurdity anything follows.

To encode the property that $X$ has no elements we use the type $X \to \emptyset$.  To encode the property that elements $a,b:X$ are not equal,
we use the type $(a=b) \to \emptyset$, and we let $a \ne b$ denote it.

Secondly, there will also be a type called $\true$, defined inductively and provided with a single element $\triv$; (the name $\triv$ comes from the word
  ``trivial'').  Its induction principle
states that, in order to prove something about (or to construct something from) every element of $\true$, it suffices to consider the special
case where the element is $\triv$.  As an example, we may prove, for any element $u : \true$, that $u=\triv$, by using induction to reduce
to proving $\triv=\triv$, a proof of which is provided by $\refl{\triv}$.  One may also prove that any two elements of $\true$ are equal by using induction twice.

There is a function $X \to \true$, for any type $X$, namely: $a \mapsto \triv$.  This corresponds, for propositions, to the statement that an
implication holds if the conclusion is true.

Thirdly, there will be a type called $\bool$, defined by induction and provided with two elements, $\yes$ and $\no$.  One may prove by induction
that any element of $\bool$ is equal to $\yes$ or to $\no$.

We may use substitution to prove $\yes \ne \no$.  To do this, we introduce a family of types $P(b)$ parametrized by a variable $b:\bool$.
Define $P(\yes) \defeq \true$ and define $P(\no) \defeq \false$.  The definition of $P(b)$ is motivated by the expectation that we will be able
to prove that $P(b)$ and $b = \yes$ are equivalent.  If there were an element $e: \yes = \no$, we could substitute $\no$ for $\yes$ in $\triv :
P (\yes)$ to get an element of $P(\no)$, which is absurd.  Since $e$ was arbitrary, we have defined a function $(\yes=\no) \to \emptyset$,
establishing the claim.

In the same way, we may use substitution to prove that successors of natural numbers are never equal to $0$, i.e., for any $n:\NN$ that $0 \ne
S(n)$.  To do this, we introduce a family of types $P(i)$ parametrized by a variable $i:\NN$.  Define $P$ recursively by specifying that $P(0)
\defeq \true$ and $P(S(m)) \defeq \false$.  The definition of $P(i)$ is motivated by the expectation that we will be able to prove that $P(i)$
and $i = 0$ are equivalent.  If there were an element $e: 0 = S(n)$, we could substitute $S(n)$ for $0$ in $\triv : P ( 0 )$ to get an element
of $P(S(n))$, which is absurd.  Since $e$ was arbitrary, we have defined a function $(0=S(n)) \to \emptyset$, establishing the claim.

We could continue defining inductively types with three elements, 
four elements, and even with $n$ elements for any $n:\NN$. It is,
however, more interesting to look at other constructions for types.

\subsection{Sum types}
\label{sec:sum-types}
There are \emph{sums} of types.  By this we mean if $X$ is a type and $Y(x)$ is a family of types indexed by a parameter $x$ of type $X$, then
there will be a type $\sum _{x:X} Y(x)$ whose elements are all pairs $(a,b)$, where $a:X$ and $b:Y(a)$. Since the type of $b$ may depend on $a$ we also call such a pair
a \emph{dependent} pair. We may refer to $X$ as the \emph{index
  type} of the sum.  

Proving something about (or constructing something from) every 
element $(a,b)$ of $\sum _{x:X} Y(x)$ is simply done for all $a:X$ and $b: Y(a)$.
Two important examples of such constructions are:
\begin{enumerate}
\item \emph{first projection} $\fst:(\sum _{x:X} Y(x)) \to X$, 
$\fst(a,b)\defeq a$;
\item \emph{second projection}
$\snd: \prod_{z: \sum _{x:X} Y(x)} Y(\fst z)$, $\snd(a,b)\defeq b$.
\end{enumerate}

Equality of two elements $(a_1,b_1),(a_2,b_2)$ of $\sum _{x:X} Y(x)$ is 
inductively defined in \cref{sec:identity-types}, like for any other type.
In the case of a structured type defined from other types, 
here $\sum _{x:X} Y(x)$ from $X$ and $Y(x)$ for any $x$,
one would like to express equality in the structured type in terms of 
the equalities in the constituent types. This explains much better
what it means for two elements of the structured type to be equal.
This is so important that we give this explanation here and below
every time we introduce a new structured type, even though not everything
can be proved yet. For complete proofs we refer to \cite{hottbook}.

An identification between two elements $(a_1,b_1),(a_2,b_2)$ of 
$\sum _{x:X} Y(x)$ means in the first place, by taking
the first projection, having an identification $i: a_1=a_2$.
We cannot expect exactly the same for the second 
projections $b_1: Y(a_1)$ and $b_2: Y(a_2)$, since they may
have different types and can therefore not be compared directly.
However, after transport of $b_1$ in the type family $Y$
along the identification $i: a_1=a_2$, a direct comparison to $b_2$
is possible. Thus we obtain that each identification $(a_1,b_1)=(a_2,b_2)$
can be viewed as a pair of an identification $i: a_1=a_2$ and an
identification $i' : \trp_{Y,i}(b_1) = b_2$. Note that the type
of $i'$ depends on $i$, so that we actually have a dependent pair:
\[
(i,i'): \sum_{i:a_1 = a_2} \trp_{Y,i}(b_1)= b_2
\]
The latter type (COULD BE LEFT OUT) can be proved to be equivalent to the
type $(a_1,b_1)=(a_2,b_2)$ in a precise sense.

\subsection{Binary products}
\label{sec:binprod-types}
There is special case of sum types that deserves to be mentioned since
it occurs quite often. Let $X$ and $Y$ be types, and consider the constant
family of types $Y(x)\defeq Y$. In other words, $Y(x)$ is a type that depends
on an element $x$ of $X$ that happens to be $Y$ for any such $x$.
Then we can form the sum type $\sum_{x:X} Y(x)$ as in the previous
section \ref{sec:sum-types}. Elements of this sum type are pairs $(x,y)$
with $x$ in $X$ and $y$ in $Y(x)\jdeq Y$. In this case the type of $y$
doesn't depend on $x$, and in this special case the sum type is called
the \emph{binary product}, or \emph{cartesian product} of the types $X$ and $Y$,
denoted by $X \times Y$.

Recall that we have seen something similar with the product type
$\prod_{x:X} Y(x)$, which we denote $X\to Y$ in case $Y(x)\defeq Y$.
The type $X \times Y$ inherits the functions $\fst,\snd$ from
$\sum_{x:X} Y(x)$, with the same definitions $\fst(x,y)\defeq x$
and $\snd(x,y)\defeq y$. Their types can now be denoted in a
simpler way as $\fst: (X \times Y)\to X$ and 
$\snd: (X \times Y)\to Y$, and they are called as before the
first and the second projection, respectively.

Again, proving something about (or constructing something from) every 
element $(a,b)$ of $X \times Y$ is simply done for all $a:X$ and $b:Y$.
An identification between two elements $(a_1,b_1),(a_2,b_2)$ of 
$X \times Y$ consists of an identification of $a_1$ and $a_2$ in $X$
and an identification of $b_1$ and $b_2$ in $Y$. In fact, 
again, the type of identifications between two pairs in $X \times Y$
is equivalent to the binary product of the types of identifications between
elements of $X$ and of identifications between elements of $Y$.

\subsection{Binary sums}
\label{sec:binsum-types}
If a sum type is of the form $\sum_{b:\bool} T(b)$, with $T(b)$
a type depending on $b$ in $\bool$, there is a simpler way of
describing it. After all, the type family $T(b)$ is fully determined
by two types, namely by the types $T(\no)$ and $T(\yes)$.
The elements of $\sum_{b:\bool} Y(b)$ are dependent pairs $(\no,a)$ with
$a$ in $T(\no)$ and $(\yes,b)$ with $b$ in $T(\yes)$. The resulting
type can be viewed as the \emph{disjoint union} of $T(\no)$ and $T(\yes)$:
from an element of $T(\no)$ or an element of $T(\yes)$ 
we can produce an element of $\sum_{b:\bool} T(b)$.  

Such types can be described more clearly in the following way.
The \emph{binary sum} of two types $X$ and $Y$, denoted $X \amalg Y$,
is an inductive type with two constructors: $\inl{} : X \to X \amalg Y$ and
$\inr{} : Y \to X \amalg Y$. Proving a property of any element of $X \amalg Y$
means proving that this property holds of any $\inl{x}$ with $x:X$ and any
$\inr{y}$ with $y:Y$. In general, constructing a function $f$ of type
$\prod_{z: X \amalg Y} T(z)$, where $T(z)$ is a type depending on 
$z$, is done by defining $f(\inl(x))$ for all $x$ in $X$
and $f(\inr(y))$ for all $y$ in $Y$.

Identification of two elements $a$ and $b$ in $X \amalg Y$ is 
only possible if they are constructed with the same constructor.
Thus $\inl(x) = \inr(y)$ is always empty, and identifications
$\inl(x) = \inl(x')$ are equivalent to identifications $x=x'$ in $X$,
and identifications
$\inr(y) = \inr(y')$ are equivalent to identifications $y=y'$ in $Y$.

\section{Equivalences}\label{sec:euivalence}


The combination of $\sum$-, $\prod$- and equality types allows
us to express important notions, as done in the following
definitions.

\begin{definition}
\label{def:contractible}
Let $X$ be a type.  The property that $X$ has exactly one element may be expressed by saying $X$ has an element such that every other element is
equal to it.  Hence it is encoded by the type $\sum_{c:X} \prod_{x:X} (c=x)$.
We call a type $X$ having this property \emph{contractible}, 
denoted by $\iscontr(X)$, and $c$ its \emph{center}. 
(Note that the notion of center depends on the proof that $X$
is contractible, rather than on the type $X$. 
Any element of a contractible type $X$ can be proved to be a center.) 
\end{definition}

An important example of a contractible type is the
\emph{singleton type} $\sum_{x:X} (a=x)$, thought of as
the subtype of the type $X$ consisting of the element $a$.
In order to see that singleton types are contractible,
take as center the element $(a,\refl{a})$. We have
to prove for any element $x$ of $X$ and identification
$e: a=x$ that $(a,\refl{a}) = (x,e)$. This we do componentwise,
as explained in \cref{sec:sum-types}. For the identification of
the first components we take (of course) $e$. For the second
components, we have to prove $\trp_{a=\_,e}\refl{a}= e$.
This follows immediately by induction on $e$.
%EASY: for $e\jdeq \refl{a}$ take $\refl{\refl{a}} : trp_{a=\_,e} = e$.

\begin{definition}
\label{def:fiber}
Given a function $f : X \to Y$ and an element $y:Y$, the \emph{fiber} (or \emph{inverse image}) $f^{-1}(y)$ consists of elements $x$ such that $f(x)
= y$.  This is encoded by defining $f^{-1}(y) \defeq \sum_{x:X} (f(x) = y)$.  In other words, an element of the fiber is a pair $(x,e)$ consisting
of an element $x$ and an element $e$ of the identity type $f(x) = y$.
\end{definition}

In set theory, a function $f : X \to Y$ is a bijection if and only if
all inverse images $f^{-1}(y)$ consist of exactly one element.
This we can also express in type theory, in an definition due
to Voevodsky. 

\begin{definition}
\label{def:equivalence}
A function $f : X \to Y$ is called an \emph{equivalence},
denoted by $\isEq(f)$,  if
\[
\prod_{y:Y} \iscontr(f^{-1}(y)).
\]
If $t:\prod_{y:Y} \iscontr(f^{-1}(y))$, 
we can define an inverse $f^{-1}: Y\to X$
by setting $f^{-1}(y) = \fst(t(y))$ and prove that $f^{-1}$ is also
an equivalence. (EXERCISE)
In such a case we say that the types
$X$ and $Y$ are \emph{equivalent}, denoted by $X\equiv Y$. 
More precisely, since there could be more than one equivalence
between two types, we define the type of equivalences from $X$ to $Y$ by
\[
(X\equiv Y) \defeq \sum_{f:X\to Y} \isEq(f) 
\]
(EXERCISE: reflexive, symmetric, transitive.)
\end{definition}
For any type $X$, the identity function $\id_X$ is an
equivalence from $X$ to $X$: for every element $a$ in $X$,
$\id_X^{-1}(a)$ is a singleton type and hence contractible.
This simple observation combined with the fact that
$\trp_{T,\refl{x}}\jdeq \id_{T(x)}$ gives us the next
lemma by induction.

\begin{lemma}\label{lem:equivalence-transport}
Let $X$ be a type, and let $T(x)$ be a type depending on $x:X$.
Then for every $x,y$ in $X$ and $e: x=y$, the function $\trp_{T,e}$
is an equivalence from $T(x)$ to $T(y)$.
\end{lemma}

\section{Universes and univalence}\label{sec:univax}

Univalence expresses, in a way made precise below, that
equivalent types are equal. This means
we are now discussing $X = Y$, and any identity type 
(see \cref{sec:identity-types})
requires an ambient type containing the subjects of the identification,
here the types $X$ and $Y$. In other words, we need a type of types.

\begin{definition}\label{def:universe}
There is type, denoted $\UU$ and called a \emph{universe}, that contains
all types discussed so far, with the exception of the types 
$X = Y$, where $X$ and $Y$ themselves are types, and of $\UU$ itself. 
\end{definition}

Using $\UU$ we can now also assign a type to type families:
if $Y(x):\UU$ for all $x:X$, $X:\UU$, then $Y$ has type $X\to\UU$.
What is the type of $X\to\UU$, and of $\UU$ itself?
The answer is: the next universe.
In this book we use, mostly implicitly, as many such universes as we need.
Note that a countable hierarchy of universes achieves
the foundational ideal of `everything has a type' while
avoiding the paradoxical $\UU:\UU$.

The univalence axiom is a statement about the universe $\UU$: 
for all types $A$ and $B$ in $\UU$ being \emph{equal} in $\UU$ is 
\emph{equivalent} to them being \emph{equivalent}. This greatly
enhances our capacity to prove that two types are equal
and to use this equality to transport properties and structure
between them. We deliberately use the term transport
here since it is precisely transport in the sense of
\cref{def:transport} along the identification
$X = Y$ stipulated by the equivalence $f$.

\begin{definition}\label{def:univalence}
Define $id_\UU$ by $\id_\UU(X)\defeq X$ for all types $X$.
We can alternatively see $\id_\UU$ as the identity function on $\UU$
and as a type family on $\UU$. Under the latter view,
given an identification $e$ of types $X$ and $Y$ in $\UU$, we can
transport elements from $X$ to elements of $Y$ using the transport
function from \cref{def:transport}:
\[
\trp_{\id_\UU, e} : X \to Y
\]
By \cref{lem:equivalence-transport}, $\trp_{\id_\UU, e}$ is an 
equivalence from $X$ to $Y$ for every $e: X = Y$. The corresponding
function from $X = Y$ to $X\equiv Y$ will be denoted by
$\trp^+_{X,Y}$. In its precise formulation,
Voevodsky's \emph{univalence axiom} (UA) postulates that $\trp^+_{X,Y}$
is an equivalence from $X = Y$ to $X\equiv Y$.
Formally: $ua : \isEq(\trp^+_{X,Y})$.
\end{definition}
The above formulation is perhaps a bit `heavy' for daily use.
Therefore we elaborate some consequences that almost always suffice.
First, we can disregard the particular inhabitant and just use that
the following principle is true under UA: 
\[
(X = Y) \equiv (X \equiv Y) \text{ for all types $X$ and $Y$.} 
\]
Second, UA implies the principle of \emph{weak univalence}:
\[
(X \equiv Y) \to (X=Y) \text{ for all types $X$ and $Y$.} 
\]
Third, if one actually uses the particular inhabitant,
one can often cast this in the following form,
where $X$ and $Y$ are types, and $f: X\to Y$ is an equivalence:
\begin{quote}
Transport from $X$ to $Y$ along the identification of $X$ and $Y$
coming from $f$ through UA is just applying $f : X\to Y$.
\end{quote}
This justifies the terminology used right after
\cref{lem:equivalence-transport}. Since this conveys an
important insight into univalence, we elaborate.
Let $p$ be a proof that $f$ is an equivalence
and consider $ua(f,p)$ witnessing that the fiber of $\trp^+_{X,Y}$
at $(f,p)$ is contractible. This means that $\trp^+_{X,Y}$
applied to the center $c \defeq \fst(ua(f,p)) : X=Y$ is equal to $(f,p)$.
Elaborating this and looking at the first component we get
precisely $\trp_{\id_\UU, c} = f$.

\begin{xca}\label{xca:C2}
   Prove that $\mathbf 2=\mathbf 2$ has exactly two elements, $\refl{\mathbf 2}$ and $\twist$ (where $\twist$ is given by univalence from the equivalence $\mathbf 2\to\mathbf 2$ exchanging the two elements of $\mathbf 2$), and that $\trans{}_{\mathbf 2,\mathbf 2,\mathbf 2}(\twist)(\twist)=\refl{\mathrm 2}$.
\end{xca}

------------------------------------------------------

\subsubsection{The univalence axiom}

\begin{definition}\label{def:typeofeq}
  Let $A$ and $B$ be types.  The \emph{type of equivalences from $A$ to $B$} is
  $$\Eq(A,B)\defequi\sum_{f:A\to B}\isEq(f).$$
If we say that $f:A\to B$ \emph{is} an equivalence, this refers to some specific $(f,p):\Eq(A,B)$.\footnote{our cavalier attitude would be easier to understand for the reader if we said that $\isEq(f)$ is a proposition}
\end{definition}
\begin{remark}\label{remark:inversetoeq}
  Equivalences have preferred ``inverses'': if $(f,p)$ is an equivalence, then for $b:B$ the element $p(b)=(c_b,q)$ is a contraction of $f^{-1}(b)$ to the \emph{center of the contraction} $c_b:f^{-1}(b)$. By the definition of the fiber, $c_b$ is a pair $(a_b,r_b)$, where $a_b:A$ and $r_b:f(a_b)=b$. If we let $g:B\to A$ be given by $g(b)\defequi a_b$, then 
$$q_{f(a)}:a=g(f(a))\oldequiv a_{f(a)}$$ and  
$$r_b:f(g(b))\oldequiv f(a_b)= b.$$

This parallels the situation for sets, where the bijections can be characterized as the functions that are one-to-one and onto.  Spelling out ``one-to-one and onto'' one gets \emph{exactly} that each fiber has a unique element, and conversely ``bijection'' means that there is an inverse function.
\end{remark}


Consider the identity $\id_{\UU}:\UU\to\UU$ of the universe itself.  Given an equality $p:A=_{\UU}B$ of types, we get a transport (c.f.~\cref{def:transport}) $\trp_{\id_{\UU},p}:A\to B$ (which we by abuse of notation later laps into also calling $p$).  In other words, transport defines a map 
$$(A=_{\UU}B)\to (A\to B).$$
We want to show that $\trp_{\id_{\UU},p}:A\to B$ is an equivalence, and by induction it is enough to do this for the case when $p$ is $\refl A$.  However, by the very definition of transport we have 
$\trp_{\id_{\UU},\refl A}(a)\defequi a$, in other words 
$$\trp_{\id_{\UU},\refl A}=\id_A.$$
Now, $\id_A$ is an equivalence, since the fiber over $a:A$ is $\sum_{x:A}(a=x)$ which contracts to $(a,\refl a)$.

% \begin{remark} % BID
%   The lack of symmetry (why \emph{left} inverse?) in this formulation is unsatisfactory and avoidable at the price of having a slightly more elaborate definition of $\Eq(A,B)$.\footnote{contractible fibers definition is also lop-sided in the sense that you define a section (left inverse) by picking the center of the contraction}
% \end{remark}

-----------------------------------------------------







\section{Propositions, sets and groupoids}
\label{sec:props-sets-grpds}

Let $X$ be a type.  The property that $X$ has at most one element may be expressed by saying that any two elements are equal. Hence it is encoded
by $\prod_{a,b:X} (a=b)$.  We shall call such a type a \emph{proposition}, and its elements will be called \emph{proofs}.

Let $X$ be a type.  If for any $x:X$ and any $y:X$ the identity type $x=y$ is a proposition, then we shall say that $X$ is a \emph{set}.
Alternatively, we shall say that $X$ is a 0-\emph{type}.\footnote{%
Sets are thought to consist of points. Points are entities of dimension 0, 
which explains why the count starts here.
One of the contributions of Vladimir Voevodsky is the extension of
the hierarchy downwards, with the notion of proposition,
including logic in the same hierarchy.
Some authors therefore call propositions $(-1)$-\emph{types}.} 

Let $X$ be a type.  If for any $x:X$ and any $y:X$ the identity type $x=y$ is a set, 
 then we shall say that $X$ is a \emph{groupoid}, also called a 1-\emph{type}.

The pattern continues.  If for any $n:\NN$, any $x:X$, and any $y:X$ 
the identity type $x=y$ is is an $n$-\emph{type}, 
then we shall say that $X$ is an $(n+1)$-\emph{type}.

We prove that every proposition is a set, from which it follows
by induction that every $n$-type is an $(n+1)$-\emph{type}.

\begin{lemma}\label{lem:prop-is-set}
Every type that is a proposition is also a set.
\end{lemma}
\begin{proof}
Let $X$ be a type and let $f: \prod_{a,b:X} (a=b)$. Let $a,b,c : X$ and
let $P(x)$ be the type $a=x$ depending on $x:X$. Then
$f(a,b):P(b)$ and $f(a,c):P(c)$. By path induction we prove for
all $q:b=c$ that $q\cdot f(a,b) = f(a,c)$. For this it suffices to
verify that $\refl{b} \cdot f(a,b) = f(a,b)$, which follows immediately.
So $q$ is equal to $f(a,c)\cdot f(a,b)^{-1}$ which doesn't
depend on $q$, so all such $q$ are equal. Hence $X$ is a set.
\end{proof}

In the following lemma we collect a number of useful results on propositions.

\begin{lemma}\label{lem:prop-utils}
Let $A$ be a type, and let $P$ and $Q$ propositions.
Let $R(a)$ be a proposition depending on $a:A$. Then we have:
\begin{enumerate}
\item\label{prop-utils-false-true} $\false$ and $\true$ are propositions;
\item\label{prop-utils-implication} $A\to P$ is a proposition;
\item\label{prop-utils-pi} $\prod_{a:A} R(a)$ is a proposition;
\item\label{prop-utils-times} $P\times Q$ is a proposition;
\item\label{prop-utils-eq} $P = Q$ is a proposition.
\end{enumerate}
\end{lemma}

\begin{proof}
(\ref{prop-utils-false-true})
If $p,q : \false$, then $p=q$ is proved by the Ex Falso rule.
If $p,q : \true$, then $p=q$ is proved by double induction,
which reduces the proof to observing that $\refl{\triv}: \triv=\triv$.

(\ref{prop-utils-implication})
If $p,q : A\to P$, then $p=q$ is proved by first observing that $p$ and $q$
are functions which, by function extensionality, are equal if they have
equal values $p(x) = q(x)$ in $P$ for all $x$ in $A$. This is
actually the case since $P$ is a proposition.

(\ref{prop-utils-pi})
If $p,q : \prod_{a:A} R(a)$ one can use the same argument as for $A\to P$
but now with \emph{dependent} functions $p,q$.

(\ref{prop-utils-times})
If $(p_1,q_1),(p_2,q_2) : P\times Q$, then $(p_1,q_1)=(p_2,q_2)$
is proved componentwise. 

(\ref{prop-utils-eq})
By UNIVALENCE, $P = Q$ is equivalent to
$(P\to Q)\times(Q\to P)$, which is a proposition by 
combining (\ref{prop-utils-implication}) and
(\ref{prop-utils-times}).
\end{proof}

Several remarks can be made here. First, the lemma supports the
use of $\false,\true$ as truth values, and the use of
$\to,\prod,\times$ for implication, universal quantification,
and conjunction, respectively. Since $\false$ is a proposition,
it follows by (\ref{prop-utils-implication}) above that
$\neg A$ as defined by $A\to\false$ is a proposition for any type $A$.
As noted before, (\ref{prop-utils-implication}) is a
special case of (\ref{prop-utils-pi}).

Notably absent in the lemma above are disjunction
and existential quantification. This has a simple reason:
$\true\amalg \true$ has the distinct elements
$\inl{\triv}$ and $\inr{\triv}$, an is therefore \emph{not} a proposition.
Similarly, $\sum_{n:\NN} \true$ has infinitely many
distinct elements $(n,\triv)$ and is not a proposition. We will see later how
to work with disjunction and existential quantification for propositions.

The lemma above has a generalization from propositions to
$n$-types which we state without proving.

\begin{lemma}\label{lem:level-n-utils}
Let $A$ be a type, and let $X$ and $Y$ be $n$-types.
Let $Z(a)$ be an $n$-type depending on $a:A$. Then we have:

\begin{enumerate}
\item\label{level-n-utils-implication} $A\to X$ is an $n$-type;
\item\label{level-n-utils-pi} $\prod_{a:A} Z(a)$ is an $n$-type;
\item\label{level-n-utils-times} $X\times Y$ is an $n$-type.
\end{enumerate}
\end{lemma}

\begin{lemma}\label{lem:isX-is-prop}
  For any type $A$, the types 
$\iscontr(A)$, $\isprop(A)$, $\isset(A)$, $\isconn(A)$, etc.\ are propositions.
(We shall only use the notation $\mathrm{isX}(A)$ for propositions about $A$.)
\end{lemma}

\begin{proof}
Recall that $\iscontr(A)$ is $\sum_{a:A} \prod_{b:A} (a=b)$.
It is clear that $A$ is a proposition if $\iscontr(A)$.
Let $(a,f)$ and $(b,g)$ be elements of the type $\iscontr(A)$.
For $(a,f) = (b,g)$ we need an $e : a=b$ and an $e' : \trp_e f = g$.
For $e$ we can take $f(a)$ and for $e'$ we invoke 
\cref{lem:prop-utils}(\ref{prop-utils-pi}).

Recall that $\isprop(A)$ is $\prod_{a,b:A}(a=b)$.
If $p$ is an element of $\isprop(A)$, then $A$ is a proposition
and hence a set by \cref{lem:prop-is-set}. Hence $\isprop(A)$
is a proposition by \cref{lem:prop-utils}(\ref{prop-utils-pi}).

We delegate the other (and future) cases to the exercises.
\end{proof}



\section{Logical operations on propositions; Propositional truncation}
\label{sec:logical-operations}

In \cref{sec:props-sets-grpds} we have seen that propositions are
closed under $\times$, $\to$ and taking products over arbitrary
index types. Moreover, $\false$ and $\true$ are propositions.
In this section we explain in detail how to do logic in type theory.
First, logical propositions are represented by the types that we
have already called propositions, that is, the types in which all
elements are equal. The reason for doing so is that the interesting
thing about a logical proposition is whether it has a proof or not.
It is therefore reasonable to require for any type representing 
a logical proposition that all its members are equal.

We have also seen that ${\amalg}$ and $\Sigma$ can lead to types
with distinct elements even though the constituents are
propositions. In order to enforce that all elements are equal
we define an operation called propositional truncation.\footnote{%
Misnomer because ... }
This is the first example of a \emph{higher inductive type}.
Higher inductive types are inductive types that also have
constructors for identifications between elements and/or
identifications between identifications.

\begin{definition}\label{def:prop-trunc}
Let $T$ be a type. The \emph{propositional truncation} of $T$
is a type  $\Trunc{T}$ defined by the following constructors:
\begin{enumerate}
\item an \emph{element} constructor $\trunc{\_} : T \to\Trunc{T}$
\item an \emph{identification} constructor $i: \prod_{x,y:\Trunc{T}} x=y$
\end{enumerate}
\end{definition}
Analogously to ordinary inductive types, any higher inductive type
comes with an induction principle that states how to prove something
about (or to construct something from) every element of the higher inductive type.
The induction principle should now also take into account 
the identification constructors. For the moment we limit
ourselves to giving a recursion principle stating how to define
a function $f$ of type $\Trunc{T} \to X$. The following data
suffices to specify such $f$:
\begin{enumerate}
\item a function $g$ of type $T\to X$ replacing $\trunc{\_}$; 
\item a function $\hat{\imath}$ of type $\prod_{x,y:X} x = y$ replacing $i$.
\end{enumerate}
These data yield a function $f$ with $f(\trunc{x})\defeq g(x)$ 
for all $x:T$. The function $\hat{\imath}$ warrants that $X$ is a
proposition so that $f$ maps equal elements of $\Trunc{T}$ to equal
elements of $X$.

Wanted:
\[
(\Trunc{A} \to B) \equiv (A \to B) \text{for $B$ a proposition}
\]



$\Trunc{A \amalg B}$
\section{Truncation}
\label{sec:truncation}

\section{Operations on sets that produce new sets}
\label{sec:operations-on-sets}

\begin{lemma}\label{lem:subtype}
Let $T$ be an $n$-type, and let $P(x)$ be a proposition depending on $x:T$. 
Then $\sum_{x:T} P(x)$ is also an $n$-type.
\end{lemma}

\begin{proof}
If $(x_1,p_1),(x_2,p_2) : \sum_{x:T} P(x)$, then by WHICH LEMMA
$(x_1,p_1)=(x_2,p_2)$ is equivalent to 
$\sum_{q:x_1=x_2} ({\trp}^P q\,p_1 = p_2)$. 
Since $P(x_2)$ is a proposition, using SEVERAL LEMMAS we get 
\[
%(x_1,p_1)=(x_2,p_2) \equiv
(\sum_{q:x_1=x_2} {\trp}^P q\,p_1 = p_2) \equiv 
(\sum_{q:x_1=x_2} \true) \equiv (x_1=x_2).
\]
Hence $\sum_{x:T} P(x)$ is an $n$-type if $T$ is.
\end{proof}
In the special case that $T$ is set we call 
$\sum_{x:T} P(x)$ a subset which we denote by 
$\set{x:T \mid P(x)}$.

\begin{lemma}\label{lem:eq_of_sets-is-set}
If $X$ and $Y$ are sets, then $X=Y$ is a set. 
In other words, $\Set$ is a groupoid.
\end{lemma}

\begin{proof}
By univalence, $(X=Y) \equiv (X\equiv Y)$. The latter type is
$\sum_{f:X\to Y} \isEq(f)$. Since $X$ and $Y$ are sets,
so is $X\to Y$ by \cref{lem:level-n-utils}. Moreover,
$\isEq(f)$ is a proposition by \cref{lem:isX-is-prop}.
It follows by \cref{lem:subtype} that $X=Y$ is a set.  
\end{proof}


% Local Variables:
% fill-column: 144
% latex-block-names: ("lemma" "theorem" "remark" "definition" "corollary" "fact" "properties" "conjecture" "proof" "question" "proposition")
% TeX-master: "book"
% End:
