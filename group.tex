%\usepackage{\xspace}
The identity type is not just any type.  In the previous sections we have seen that the identity type $a=_Aa$ reflects the ``symmetries'' of a term $a$ in a type $A$.  Symmetries have special properties; for instance you can rotate a square by $90^o$, and you can rotate it by $-90^o$, undoing the first rotation.  
Symmetries can also be composed, and this composition respects certain rules that holds in all examples.  When coining the concept of ``symmetries'', we should isolate these common rules for all our examples, but also show, conversely, that anything satisfying these rules actually {\em is} an example.  This is the purpose of the mathematical term ``group''. 

As an instance of a property that holds in some examples but not in others, we have seen that sometimes the order in which we use our symmetries matters, and sometimes it does not, see \refer{}.  Hence, the concept of a group should not have a rule allowing you to change the order arbitrarily.

In Section~\refer \footnote{((can we label everything so that cross-referencing can be done continuously?))} we saw that the identity type was ``reflexive, symmeric and transitive''.  

With inspiration of geometric and algebraic origins, it became clear to mathematicians at the end of the 19'th century that the properties of such symmetries could be codified by saying that they form a {\em group}. 

The examples Klein and Lie were interested in were of a type making it admissible to say that a group is the identity type $a=_Aa$ for {\em some} type $A$ and element $a:A$.  
However, in elementary texts it is customary to restrict the notion of a group to the case when $a=_Aa$ is a {\em set} as in Section~\refer{}.  This makes some proofs easier, since we given two elements $g,h:a=_Aa$, then the identity type $g=h$ is a proposition, \ie $g$ can be equal to $h$ in at most one way.  hence questions relating to uniqueness will never be a problem.

  % Conversely, once one has given an abstract definition of a group it emerges that they all are identity types (see Lemma~\refer{}).  This chapter is about groups, and their basic properties; everything from the point of view of identity types which arguably is closer to the geometric origins of group theory than the abstract approach which is usual in traditional textbooks.  

See the end of the chapter \ref{sec:grouphistory} for a brief summary of the early history of groups.  
\begin{remark}
  The reader may wonder about the status of the identity type $a=_Aa'$ where $a,a':A$ are different elements.  One problem is of course that there is no obvious ways of composing and another is that $a=_Aa'$ does not have a distinguished element such as $\mathrm{refl{}_a}:a=_Aa$.
Given $f:a=_Aa'$ we can use transport along $f$ to compare $a=_Aa'$ with $a=_Aa$ (much as affine planes can be compared with the standard plane or a finite dimensional real vector space is isomorphic to some Euclidean space), but absent existence and choice of such an $f$ the identity types $a=_Aa'$ and $a=_Aa$ are very different objects.
\end{remark}


\begin{remark}
  When considering the identity type $a=_Aa$, the terms $x:A$ with $x$ unequal to $a$ are irrelevant, and we are free to consider only the {\em connected} case, \ie where $x=_Aa$ is always inhabited.
\end{remark}


With this established, we let the {\em type} of groups be defined as follows:

\begin{definition}\label{def:typegroup}\footnote{we must define  $\isset$ and propositional truncation.  Alternatively we must define $\isonetype$ and $\conn$}
  The {\em type of groups} is
%$$\typegroup=\sum_{A:\mathcal U}A\times\isonetype(A)\times \conn_0A.$$
$$\typegroup\defequi\sum_{A:\mathcal U}\sum_{a:A}\isset(a=_Aa)\times\prod_{x:A}||x=_Aa||.$$
We refer to a term in $\typegroup$ as a {\em group}.  A group $(A,a,p,q):\typegroup$ will be referred to simply as $\aut_A(a)$.
\end{definition}
\begin{remark}\label{rem:aut}
The abuse inherent in writing $\aut_A(a)$ instead of $(A,a,p,q)$ is innocent: since $\isset(a=_Aa)\times\prod_{x:A}||x=_Aa||$ is a proposition  \refer\footnote{where do we explain that this is a proposition?}  the witnesses $(p,q)$  is unique. 
\end{remark}
\begin{example}
  Since any pointed connected $1$-type represent a group, there is no shortage of examples, but perhaps it is worth while to consider some specially designed examples.
  \begin{enumerate}
  \item Recall that $1$ has the single element $*$.\footnote{what is the notation/reference for our chosen final element?} Then $\aut_1(*)$ is a group called the {\em trivial group}.
  \item More generally, if $S$ is a set, we're after a pointed connected $1$-type $(A,a)$ so that $a=_Aa$ model all the permutations of $S$, \ie $S=_{\Set}S$.  The only thing wrong about the type $\Set$ (apart from it being large\footnote{how do we deal with that?}) is that it is not connected.  So we {\em pick
 the component of $\Set$ containing $S$, pointed at $S$}!\footnote{it's so simple -- so very simple -- that only a child can do it!}  The component of $S$ is picked out by considering 
$$A=\Set_{(S)}\defequi\sum_{X\in\Set}||S=X||.$$  The connected $1$-type $\Set_{(S)}$ is pointed at $S$ (and the fact that $S=S$ is nonempty since $\refl S:S=S$).    Then $(S=_{\Set_{(S)}}S)=(S=_{\Set}S)$ (both the first and the second projections must be equal in a $\Sigma$-type, but in this case the second projection is a (true) proposition).  So, the {\em group of permutations of $S$} is defined to be $(\Set_{(S)},S)$.
  \end{enumerate}
\end{example}

\begin{remark}
In Lemma~\ref{lem:idtypesgiveabstractgroups} we will see that groups satisfy a set of laws justifying the name ``group''
%we may associate an abstract group $(a=_Aa,e,{-}^{-1},\cdot)$ 
and we will later show that groups are uniquely characterized by these laws.
\end{remark}
\begin{remark}
  The $\isset(a=_Aa)$ is sometimes more of a nuisance, and deleting it gives the simpler concept of \aninftygp, see Definition~\ref{sec:inftygps}.
\end{remark}

\begin{xca}
   Let $\aut_A(a):\typegroup$ and let $b:A$.  Prove that $\aut_A(a)=\aut_A(b)$.  Likewise for \inftygps when you get that far.
\end{xca}
\begin{remark}\label{rem:monoidandabsgplarge}
 In Definition~\ref{def:typegroup} the first $\sum$ in $\typegroup$ ranges over the entire universe $\mathcal U$.  Hence, $\typegroup$ does not belong to $\mathcal U$, but rather to the next universe. \footnote{where do we explain this?}  This tendency that the ``type of all the types we are interested in'' is a ``large type'' is a regular feature of the theory and since it will not cause any trouble for us, we will not be consistent in pointing it out in the future.
  \end{remark}

\section{The identity type as an abstract group }
Studying the identity type leads one to the definition of what a group should be:
Let $A$ be a type, and for the moment let $a=b$ be shorthand for $a=_Ab$.  In Section~\refer{} we saw that
\begin{enumerate}
\item[R] {\bf Reflexivity.} For any $a:A$ there is an element 
$$\refl a{}:a=a$$ (by definition)
\item[S] {\bf Symmetry.} For any $a,b:A$ there is a an element $$\symm{}_{a,b}:(a=b)\to (b=a)$$ defined by $\symm{}_{a,a}(\refl a{})\defequi\refl a{}$
\item[T] {\bf Transitivity.} For any $a,b,c:A$ there is an element $$\trans{}_{a,b,c}:(b=c)\to((a=b)\to(a=c))$$ defined by $\trans{}_{a,a,a}(\refl a{})(\refl a{})\defequi \refl a{}$.
\end{enumerate}
\footnote{\em\bf I have swapped the order of the input in trans so that it can fit.  I know you hate it and will force me to recant}

 To emulate classical notation, for fixed $a:A$,  for the moment let's write 
 \begin{enumerate}
 \item $G$ instead of $a=_Aa$,
 \item $e$ instead of $\refl a{}$
 \item $g^{-1}$ instead of $\symm_{a,a}(g)$, when $g:G$
 \item $g\cdot h$ instead of $\trans_{a,a,a}(g)(h)$ when $g,h:G$.
 \end{enumerate}
 What properties can we show about this without knowing anything about $A$ and $a$? For convenience, here is a list of the results we are aiming for: for all $g,g_1,g_2,g_3:G$ we will construct elements in all the following types 
 \begin{enumerate}
 \item $g=g\cdot e$ \footnote{redundant (keep)}
 \item $g=e\cdot g$
 \item $g^{-1}\cdot g=e$
 %\item $g\cdot g^{-1}=e$ redundant (remove)
% \item $(g^{-1})^{-1}=g$ redundant (remove)
 \item $g_1\cdot(g_2\cdot g_3)=(g_1\cdot g_2)\cdot g_3$.
 \end{enumerate}
 \begin{remark}
   One should worry about many things when one sees this list.  For instance, for the particular case of $g$ being $e$, are the elements in $e=e\cdot e$ given in the first and second item equal?  Since $G$ is a set, such worries become irrelevant: $e=e\cdot e$ is then a proposition, so any two inhabitants are equal.
 \end{remark}


We do $g=e\cdot g$ in some detail (remember that is shorthand for $e=\refl a{}$)
\begin{definition}\label{def:p1}
  Let $A$ be a type and $a, b:A$ and $g:a=b$ be elements.  Then $p_1(a,b,g):g=_{a=b}g\cdot e$ is the element defined by induction by saying that $p_1(a,a,e)$ is $\refl e:e=e\cdot e$.
\end{definition}
\begin{remark}
  This makes sense since we {\em defined} $e\cdot e\defequi e$ (or, as it was originally formulated, $\trans_{a,a,a}(\refl a{})(\refl a{})\defequi \refl a{}$).  We'll say that we produce $p_1(a,b,g)$ by ``induction on $b$'', the case where $b$ is $a$ (and $g$ is $e$) is the start of the induction; the induction priciple for the identity type $a=b$ then finishes the construction.

As constructed, $p_1$ is actually a term in the type 
$$\prod_{a:A}\prod_{b:A}\prod_{g:a=b}(g=g\cdot e)$$ -- it is constructed ``uniformly'' or ``naturally'' for all $a,b,g$: think of it as a function with $(a,b,g)$ as input and $p_1(a,b,g):g=g\cdot e$ as output.  

To add a little meat on the definition of $p_1$: in the definition of the identity type, for each $a:A$ let $P$ be the type family given by $P(b,g)\defequi (g=g\cdot e)$ for each $b:A$ and $g:a=b$.  According to the definition of the identity type, in order to produce elements in $P(b,g)$ for arbitrary $b$ and $g$ it suffices to give an element in $P(a,e)\oldequiv (e=e\cdot e)\oldequiv (e=e)$, and $\refl e$ will do.\footnote{we have chosen to give the identity type such that the naturality in $a$ is not explicit}
\end{remark}
\begin{definition}\label{def:p3}
  Let $A$ be a type and $a,b:A$ and $g:a=b$ elements.  Then $p_3(a,b,g):g^{-1}\cdot g=_{a=_Aa} e$ is the element defined by induction by saying that $p_3(a,a,e)$ is $\refl e:e=e\cdot e$ [which makes sense since $e^{-1}\equiv e$ and $e\cdot e\equiv e$].
\end{definition}
% \begin{definition}
%   Let $A$ be a type and $a,b:A$ and $g:a=b$ elements.  Then $p_5(a,b,g):(g^{-1})^{-1}=_{a=_Aa} e$ is the element defined by induction by saying that $p_5(a,a,e)$ is $\refl e:(e^{-1})^{-1}=e$ [which makes sense since $e^{-1}\equiv e$].
% \end{definition}
\begin{definition}\label{def:p4}
  Let $A$ be a type and $a,b,c,d:A$ and $g_3:a=b$, $g_2:b=c$ and $g_1:c=d$ elements.  Then $p_4(a,b,c,d,g_1,g_2,g_3):g_1\cdot(g_2\cdot g_3)=_{a=_Ad}(g_1\cdot g_2)\cdot g_3$ is the element defined by induction by saying that $p_4(a,a,a,a,e,e,e,e)$ is $\refl e:e\cdot(e\cdot e)=(e\cdot e)\cdot e$ [which makes sense since $e\cdot e\equiv e$].
\end{definition}
\begin{remark}
  This last definition is somewhat more complicated than the others, in the sense that in order to unravel the induction to exactly the form accepted in the definition of the identity type, we need to apply the rule three times.  ((write out))
\end{remark}

\begin{xca}\label{xca:p2}
    Define $p_2(a,b,g)$ %and $p_4(a,b,g)$ 
by the exactly the same procedure, completing the list.
\end{xca}


These properties of the identity type are bundled together in the concept of an abstract group, under the additional hypothesis that we are dealing with a set.

  \begin{definition}\label{def:abstractgroup}
    An {\em abstract group} is a set $G$ together with
\begin{enumerate}
\item an element $e:G$, 
\item a binary operation taking a pair of elements $g_1,g_2:G$ to a third element which we call $g_1\cdot g_2:G$ such that
  \begin{enumerate}
  \item %$e$ is a ``neutral element'': 
if $g:G$, then $g\cdot e=e\cdot g=g$ and
  \item %satisfying ``associativity'': 
if $g_1,g_2,g_3:G$, then 
$$g_1\cdot(g_2\cdot g_3)=(g_1\cdot g_2)\cdot g_3,$$
  \end{enumerate}
\item %inverses: 
to every $g:G$ there is a $g^{-1}:G$ such that $%g\cdot g^{-1}=
g^{-1}\cdot g=e$.
\end{enumerate} 
We refer to $e$ as the {\em unit element}, $g_1\cdot g_2$ as the {\em product of $g_1$ and $g_2$} and $g^{-1}$ as the {\em inverse of $g$}.  The {\em unit laws} will then be $g\cdot e=e\cdot g=g$, the {\em associativity law} is $g_1\cdot(g_2\cdot g_3)=(g_1\cdot g_2)\cdot g_3$ and $%g\cdot g^{-1}=
g^{-1}\cdot g=e$ is referred to as the {\em law of inverses}. 
  \end{definition}

In conclusion we have proved that groups ``are'' abstract groups:

  \begin{lemma}\label{lem:idtypesgiveabstractgroups}
    If $A$ is a $1$-type (alternatively called a ``type of level at most $3$'' in \refer) and $a:A$ is an element, then $a=_Aa$, together with $e=\refl a{}$, $g^{-1}=\symm_{a,a}g$ and $g\cdot h=\trans_{a,a,a}(g)(h)$ define an abstract group 
$$(a=_Aa,e,{-}^{-1},\cdot).$$
  \end{lemma}
  \begin{proof}
    The elements $p_1,\dots p_4$ of Definitions~\ref{def:p1}, \ref{def:p3} and \ref{def:p4} and
Exercise~\ref{xca:p2} show that all the relevant identity types (which are propositions since $A$ is a $1$-type) are inhabited as required.
  \end{proof}

  \begin{remark}
    It is handy to break up the rather long Definition~\ref{def:abstractgroup}  by saying that the first two points (\ie the presence of the unit element and the product satisfying the unit and associative laws) define a {\em monoid} and if we in addition have inverses satisfying the law of inverses, then we have an abstract group.
    \end{remark}

  
    \begin{remark}\label{rem:typemonoidabstrgp}
        Summing up in language (only) a machine can handle, the {\em type of monoids} is
$$\typemonoid\defequi \sum_{M:\mathcal U}\sum_{e:M}\sum_{\mu{}:M\to M\to M}
\isset{(M)}\times\mathrm{Monoidlaws}(G,e,\mu)
$$
where 
$$\mathrm{Monoidlaws}(M,e,\mu)\defequi\mathrm{Unitlaws}(G,e,\mu)\times\mathrm{Assoclaw}(G,\mu{})$$and
\begin{align*}
  \mathrm{Unitlaws}(G,e,\mu)\defequi\prod_{g:G}
&(g=\mu{}(g)(e))\times(g=\mu{}(e)(g)),\\
\mathrm{Assoclaw}(G,\mu{})\defequi\prod_{g_1,g_2,g_3:G}&\mu{}(g_1)(\mu{}(g_2)(g_3))=\mu{}(\mu{}(g_1)(g_2))(g_3).
\end{align*}
In the human language we used above, $\mu(g)(h)=g\cdot h$ and $\iota(g)=g^{-1}$ and $\mathrm{Unitlaws}$ and $\mathrm{Assoclaw}$ spell out to the machine that the unit behaves like a unit and that the multiplication is associative.  
The
{\em type of abstract groups} is
$$\typeabsgp\defequi
\sum_{(G,e,\mu):\typemonoid}\sum_{\iota\colon G\to G}\prod_{g:G}(\mu{}(\iota{}(g))(g)=e).$$
% where 
% $$\mathrm{Grouplaws}(G,e,\mu,\iota)\defequi\mathrm{Monoidlaws}(G,e,\mu)\times \mathrm{Invlaws}(G,\iota{},\mu{},e)$$
% and
% $$\mathrm{Invlaws}(G,e,\mu{},\iota{})\defequi
% \prod_{g:G}(\mu{}(\iota{}(g))(g)=e)\times
% (\mu{}(g)(\iota{}(g))=e)\times
% (\iota{}(\iota{}(g))=g).$$
We will typically refer to a monoid as a triple $(M,e,\mu)$ ommitting the names for the (true) $\isset$ and unit and associativity laws and likewise, an abstract group wil be referred to as a quadruple $(G,e,\mu,\iota)$.
\end{remark}
  \begin{remark}
Without the demand that the underlying type of an abstract group or monoid is a set, life would be more complicated.  For instance, for the case when $g$ is $e$, the unit law of Definition~\ref{def:abstractgroup} (or alternatively $\mathrm{Unitlaws}(G,\mu{},e)(e)$ in Remark~\ref{rem:typemonoidabstrgp}) would provide {\em two} (potentially different) proofs that $e=e\cdot e$ and we would have to separately insist that they agree.  This problem vanishes in the setup we adopt below for \inftygps.
  \end{remark}

  \begin{xca}
    For an element $g$ in an abstract group $(G,e,\mu,\iota)$, prove that $g\cdot g^{-1}=e$ and $(g^{-1})^{-1}=g$ (for the machines among us: show that both the propositions
$
(\mu{}(g)(\iota{}(g))=e)$ and $
(\iota{}(\iota{}(g))=g)$ are nonempty).
  \end{xca}


\section{Homomorphisms}
\label{sec:hom}


The notion of a group homomorphism from $G=\aut_A(a)$ to $H=\aut_B(b)$ is simple: it is an arrow $f:A\to B$ that ``sends $a$ to $b$'', \ie together with a $p:a=_Bf(b)$:
\begin{definition}\label{def:grouphomomorphism}
  The type of {\em group homomorphisms} from $G=\aut_A(a):\typegroup$ to $H=\aut_B(b):\typegroup$ is defined to be
$$\Hom(G,H)=\sum_{f:A\to B}f(a)=_Bb.
$$
\end{definition}
This should be contrasted with the usual notion of a group homomorphism $f\colon G\to H$ of abstract groups where we must specify that in addition to preserving the neutral element ``$f(e_G)=e_H$'' it must preserve multiplication: ``$f(g)\cdot_H f(g')=f(g\cdot_G g')$ (where we have set the name of the group as a subscript to $e$ and $\cdot$).  In our setup this is simply true:

\begin{definition}\label{def:grouphomomaxioms}
  ((prove/define the standard axioms))
\end{definition}


\section{\inftygps}\label{sec:inftygps}
Disregarding the set-condition we get the simpler notion of \inftygps:
\begin{definition}
  $$\typeinftygp\defequi \sum_{A:\mathcal U}\sum_{a:A}\prod_{x:A}||x=_Aa||.$$
\end{definition}

\begin{remark}\label{rem:pointedtypes}
  In the literature it is not uncommon to refer to the terms in $\typegroup$ as ``pointed, connected groupoids'', but from our geometric perspective through symmetries it is not unreasonable that we simply call them ``groups''.  Likewise, \inftygps are justifiably also known as pointed connected types;  the type of {\em pointed types} being
$$\pttype\defequi\sum_{A:\mathcal U}A,$$
and given two pointed types $(A,a)$ and $(B,b)$, the type of {\em pointed maps} from $(A,a)$ to $(B,b)$ is
$$((A,a)\to_*(B,b))\defequi\sum_{f\colon A\to B}f(a)=b.$$
\end{remark}

  
\footnote{Let $\typeset\defequi \sum_{A:\mathcal U}\isset(A)$.}
\begin{definition}\label{def:classifyingspace}
  If $G\oldequiv\Aut_A(a):\typeinftygp$, then the underlying pointed type $BG\defequi (A,a)\colon\pttype$ is called the  {\em classifying space}.  We retain the same language also for ordinary groups in which case the classifying space is a $1$-type.   For \inftygps the definition is identical.
\end{definition}
\begin{remark}
  In view of Remark~\ref{rem:pointedtypes} and Definition~\ref{def:classifyingspace} we see that given groups (or even \inftygps) $G$ and $H$ we have a definitional equality
$$\Hom(G,H)\oldequiv(BG\to_*BH).$$
Generically, if $G$ is \aninftygp, we let $\pt_G:BG$ (and sometimes simply $\pt$ if $G$ is clear from the context) be the distinguished point (so that $G\oldequiv\aut_{BG}(\pt_G)$).
\end{remark}


  


\section{$G$-sets}
\label{sec:gsets}
One of the goals of this section is to prove that, in a precise sense, any abstract group is a group.  In doing that, we are invited to explore how abstract groups should be thought of as symmetries and introduce the notion of a $G$-set.  However, this takes a pleasant detour where we have to explore the most important feature of groups: the {\em act} on things (giving rise to symmetries)!

Before we handle the more complex case of abstract groups, let us see what this looks like for groups.

\begin{definition}
  For $G$ a group (or \inftygp), a {\em $G$-type} is a function 
  $$X\colon BG\to\mathcal U$$\footnote{in order for this to type check $BG$ must be considered here to be an unpointed type... should we care?}
and $X(\pt_G)$ is referred to as the {\em underlying type}.
If the underlying type is a set, then $X$ is called a {\em $G$-set}.  
Otherwise said, the type of $G$-types is $\Type_G\defequi(BG\to\mathcal U)$ and the type of $G$-sets is $\Set_G\defequi(BG\to\Set)$.
%$$\Type_G\defequi (BG\to\mathcal U),\qquad \Set_G\defequi (BG\to\Set).$$
\end{definition}

\begin{example}\label{def:principaltorsor}
  If $G$ is a group (or \inftygp), then 
$$\princ_G(z)\defequi(\pt_G=z)$$ is a $G$-set (or $G$-type) called the {\em principal $G$-set (or type)}.\footnote{what do we call this $G$-type?}
\end{example}

\begin{remark}
  A $G$-type $X$ is often presented by focusing on the underlying type $X(\pt_G)$  and providing is with a structure relating it to $G$ determining the entire function $X\colon BG\to\mathcal U$. 
\end{remark}

\begin{xca}
  Use that $BG$ is connected to show that if $X$ is a $G$-set, then $X(z)$ is a set for all $z:BG$ (\ie $\prod_{z:BG}\isset(X(z))$).
\end{xca}
\begin{definition}
  Given a group (or \inftygp) $G$, the type of {\em$G$-torsors} (or {\em principal homogeneous $G$-types}) is  
$$\typetorsor_G\defequi\sum_{X:\Type_G}||X=\princ_G||,$$
where $\princ_G$ is the principal $G$-torsor of Example~\ref{def:principaltorsor}. 
\end{definition}
\begin{remark}
  Note that if $G$ is a group, then $\princ_G$ is a $G${\em-set}, and so $||X=\princ_G||$ will be empty unless $X$ is a $G$-set too, and so in this case we could more simply have said $\typetorsor_G\defequi\sum_{X:\Set_G}||X=\princ_G||.$  Observe that for a group $G$, $\typetorsor_G$ is a connected $1$-type (admittedly in a higher universe) and so -- by specifying the base point $\princ_G$ -- it represents a group!  Guess which one:
\end{remark}
\begin{lemma}
  If $G$ is a group (or \inftygp), then the pointed type $(\typetorsor_G,\princ_G)$ is equal to to $(BG,\pt_G)$,\footnote{in the appropriate universe} so both represent $G$. 
\end{lemma}

\begin{proof}
  ((write))
\end{proof}

We use this as our inspiration for trying to construct a group from an abstract group.  We define totally analogously the type of torsors for an abstract group.  It will then be a relative simple matter to show that the processes of picking the underlying abstract group of a group and taking the group represented by the torsors of an abstract group are inverse to each others.
\begin{enumerate}
\item define $G$-type/set
\end{enumerate}
\subsubsection{$G$-torsors}
\label{sec:Gtorsors}
\begin{definition}
  A $G$-torsor is a $G$-set which is isomorphic to the underlying $G$-set of $G$ (write out - avoid conflict of notation wrt $|G|$)
\end{definition}
\begin{example}
  If $\aut_A(a)\colon\typegroup$ with underlying abstract group $G$, then for any $b:A$ the set $a=_Ab$ has a natural structure of a $G$-torsor as follows ((write))
\end{example}

\footnote{how deeply do we want to integrate univalence?}
\begin{lemma}
  \label{lem:Groupsareidentitytypes}Let $G$ be an abstract group.  Then the abstract group associated with $\aut_{\Gtorsor}(|G|):\typegroup$ is equal to $G$ in the type of abstract groups.

Conversely, given $\aut_A(a):\typegroup$, let $G$ be the associated abstract group.  Then $A=\Gtorsor:\mathcal U$ and $\aut_A(a)=\aut_{\Gtorsor}(|G|):\typegroup$.
\end{lemma}
In essence we have shown that our ``abstract group'' is indeed the group of symmetries of something.

\section{structure of identity types}
\section{automorphism 1-group = fundamental group (hint at higher groups)}
\section{homomorphisms induced by functions (early)}
\section{more examples: symmetric groups, integers, cyclic groups and modular arithmetic}
\section{group actions, orbits and fixed points}
\section{subgroups}
\section{Cayley's theorem}
\section{Historical remarks}\label{sec:grouphistory}

% \begin{remark}
%   Notice that the last statement  (``More precisely\dots'')  not only asserts that there {\em exist} inverses, but that there actually is a (preferred and consistent) way to produce them.  

% Classically this was in many instances unnecessay to say because there was a unique inverse, and the distinction is not mentioned in introductory texts.  However, then this very point had to be revisited later on.  In our proof relevant setting it is obvious that the ultimate statement will have to go beyond an assertion that inverses exist.
% \end{remark}
