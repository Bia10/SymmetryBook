\documentclass[a4,12pt]{amsart}

\newcounter{chapter}% HACK since amsart has no chapter counter like amsbook
%and there is \numberwithin{theorem}{chapter} in macros.tex,
%seemingly overruled by \numberwithin{theorem}{section} in top.tex

\input macros
\begin{document}
\input topZTors

\section{The integers}
\label{sec:integers}

We define the type of integers in one of the many possible ways.

\begin{definition}\label{def:integers}
Let $\zet$ be the inductive type with the following three constructors:
\begin{enumerate}
\item $\zzero: \zet$ for the integer number zero, 
$0 \defeq \zzero$
\item $\zpos: \NN \to \zet$ for positive numbers,
$1 \defeq \zpos(0),\ldots$.
\item $\zneg: \NN \to \zet$ for negative numbers, 
$-1 \defeq \zneg(0),\ldots$
\end{enumerate}

The \emph{embedding} function $i:\NN\to\zet$ is defined by induction,
setting $i(0)\defeq \zzero$, $i(S(n))\defeq \zpos(n)$.
Like the type $\NN$, the type $\zet$ is a set with decidable equality
and ordering relations,
and we denote its elements often in the usual way as $\ldots,-1,0,1,\ldots$.

One well-known equivalence is \emph{negation} ${-}:\zet\to\zet$, 
also called \emph{complement}, inductively defined by setting 
$-\zzero\defeq \zzero$, 
$-\zpos(n)\defeq \zneg(n)$, 
$-\zneg(n)\defeq \zpos(n)$.
Negation is its own inverse.

The \emph{successor} function $s:\zet\to\zet$ is defined inductively setting 
$s(\zzero)\defeq \zpos(0)$, 
$s(\zpos(n))\defeq \zpos(S(n))$,
$s(\zneg(n))\defeq -i(n)$. For example, we have
$s(-1)\jdeq s(\zneg(0))\jdeq -i(0) \jdeq \zzero \jdeq 0$.
By induction on $n:\NN$ one proves $s(i(n))=i(S(n))$, 
so that one can say that $s$ extends $S$
on the $i$-image of $\NN$. 

The successor function $s$ is an equivalence.
It is instructive to depict iterating $s$ in both directions as 
a doubly infinite sequence containing all integers:
\[
\ldots \mapsto \zneg(1) \mapsto \zneg(0) \mapsto \zzero \mapsto \zpos(0) \mapsto \zpos(1) \mapsto \ldots
\]

The inverse $s^{-1}$ of $s$ is called the \emph{predecessor} function.
We denote the $n$-fold iteration of $s$ as $s^n$, and
the $n$-fold iteration of $s^{-1}$ as $s^{-n}$.

Addition of integers is defined inductively by setting
$z + \zzero\defeq z$, 
$z + \zpos(n)\defeq s^{n+1}(z)$, 
$z + \zneg(n)\defeq s^{-(n+1)}(z)$.
Again, addition extends $+$ on the $i$-image of $\NN$,
see \cref{xca:addition-on-Z-and-N}. 
From addition and unary $-$ one can define a binary
\emph{substraction} function setting $z-y \defeq z+(-y)$.
\end{definition}

\begin{xca}\label{xca:addition-on-Z-and-N}
Show that $i(n+m)=i(n)+i(m)$ for all $n,m:\NN$.
\end{xca}


\section{The integers as a group}
\label{sec:integers-group}

In the previous section we first defined the \emph{set}
of integers $\zet$. Then we defined functions that add
algebraic stucture to $\zet$ (the structure of a free group 
with one generator). In this section we
show that this structure can also be obtained from the
symmetries of a certain element in a certain type.
This is one of the keypoints of this book, 
therefore pointed out early on.

\begin{lemma}\label{lem:one-orbit-int}
Let $s$ be as in \cref{def:integers}, and 
let $f:\zet\to\zet$ be such that $f\circ s = s\circ f$. 
  \begin{enumerate}
  \item\label{item-one-orbit} For all $z:\zet$ there is a unique $n:\NN$
such that either $z=s^{-(n+1)}(0)$, or $z=s^{n}(0)$.
  \item\label{item-f(0)-nonneg} For all $n:\NN$, if $f(0)=s^{n}(0)$, then $f=s^{n}$.
  \item\label{item-f(0)-nonpos} For all $n:\NN$, if $f(0)=s^{-n}(0)$, then $f=s^{-n}$.
  \end{enumerate}
\end{lemma}
\begin{proof}
From $f\circ s = s\circ f$ we get $f\circ s^n = s^n\circ f$
and $f\circ s^{-n} = s^{-n}\circ f$ by induction on $n:\NN$.

(\ref{item-one-orbit}) Induction on $n:\NN$ proves $s^{n}(0)=n$, 
as well as $s^{-n}(0)=-n$. Uniqueness is easy.

(\ref{item-f(0)-nonneg}) Assume $f(0)=s^{n}(0)$.  
Given $z:\zet$, let $m$ be such that either $z=s^{m}(0)$, 
or $z=s^{-(m+1)}(0)$. In the first case we calculate
\[
f(z)=f(s^{m}(0))=s^{m}(f(0))=s^{m}(s^{n}(0))=s^{n}(s^{m}(0))= s^{n}(z),
\]
so that $f=s^{n}$ by function extensionality. 
The second case is very similar;
(\ref{item-f(0)-nonpos}) goes like (\ref{item-f(0)-nonneg}).
\end{proof}

\begin{corollary}\label{cor:pre-torsor-int}
We have $\zet\equiv \sum_{f:\zet\to\zet} (f\circ s = s\circ f)$.
\end{corollary}
\begin{proof}
First observe that $f\circ s = s\circ f$ is a true proposition
for all $f=s^n$ and $f=s^{-n}$. Recall that proofs of propositions
may be left out from dependent pairs. Thus we
define $e : \zet\to \sum_{f:\zet\to\zet} (f\circ s = s\circ f)$ 
inductively by setting 
$e(\zzero)\defeq \id_\zet$, 
$e(\zpos(n))\defeq s^{n+1}$,
$e(\zneg(n))\defeq s^{-(n+1)}$.
By \cref{lem:one-orbit-int}, $e$ is a well-defined equivalence.
\end{proof}

Again by \cref{lem:one-orbit-int}, if $f\circ s = s\circ f$,
then $f: \zet\to\zet$ is an equivalence. 
Using UA, we get by \cref{cor:pre-torsor-int} the equivalence
\[
\zet\equiv \sum_{f:\zet=\zet} (f\circ s \circ f^{-1} = s).
\]
Recall that $f\circ s \circ f^{-1}$ is transport of $s$ by
conjugation with $f$. Using the characterization of equality 
in $\Sigma$-types from \cite[Theorem 2.7.2]{hottbook} we get the equivalence
\[
\zet\equiv ((\zet,s) = (\zet,s)).
\]
One particular equivalence is $e$ from the proof
of \cref{cor:pre-torsor-int}, with inverse $e^{-1}$.
We have $e^{-1}(\id_\zet) = 0$.
The type $(\zet,s) = (\zet,s)$ of symmetries of $(\zet,s)$
has a natural algebraic structure called path-algebra
induced by composing and taking inverses (of paths,
and modulo UA also of equivalences), and $\refl{(\zet,s)}$
(the identity equivalence).
%if $f,g:(\zet,s) = (\zet,s)$, then $f\circ g:(\zet,s) = (\zet,s)$.
%Moreover, elements of $(\zet,s) = (\zet,s)$ have inverses.
This algebraic structure is transported by $e^{-1}$ to $\zet$
and gives exactly the group structure defined by ${+},{-},0$.

One important issue has been ignored up to now:
What is the type of $(\zet,s)$? 
One possible answer is: $\sum_{X:\UU}(X=X)$.
The following exercise shows that we do not get the 
property that $(X,f)=(X,f)$ is equivalent to
$(\zet,s) = (\zet,s)$ for all $(X,f) : \sum_{X:\UU}(X=X)$.
It is for this reason that in the next section
another choice will be made.

\begin{xca}\label{xca:zet-symmetries}
Figure out the symmetries of $(\zet,\id_\zet)$ (easy) and 
of $(\zet,s^2)$ (hard).
\end{xca}

\section{Z-Torsors}\label{sec:ZTorsors}

In this section, pairs $(X,f)$ have type $\sum_{X:\UU}(X\to X)$
without explicitly mentioning this. Moreover, terms of
propositional type that do not interest us are denoted by $!$,
or even left out alltogether. Nested pairs may be written as tuples.

\begin{definition}\label{sec:ZTors}
The pointed type of $\zet$-torsors is defined by
\[
\TorZ\defeq \sum_{(X,f)}\Trunc{(X,f)=(\zet,s)}) \text{~with~}
\pt\defeq ((\zet,s),\trunc{\refl{(\zet,s)}}).
\]
\end{definition}
Note that $\Trunc{(X,f)=(\zet,s)}$ has propositional consequences
such as $\isset(X)$, $s\circ f = f \circ s$  and $\isEq(f)$.

The aim of this section is to show that $\TorZ$ has the
universal property of the circle. Fix $A:\UU$, $a:A$ and $p: a=_A a$. 
We are going to define types and
functions depending on this input, and will use $p$ in various
denotations to express this dependence. 

\begin{definition}\label{def:TBN}
For every $(X,f)$, define
\begin{align*}\label{eq:TBD}
P_p(X,f)&\defeq \sum_{a':A}~\sum_{h:X\to a=a'}~\prod_{x:X} h(f(x))=p\ct h(x).
\end{align*}
\end{definition}

\begin{lemma}\label{lem:guided-null-hmtps}
For every $(X,f,t):\TorZ$, the type $P_p(X,f)$ is contractible
with center $c_p(X,f,t)$ defined in the proof.
\end{lemma}
\begin{proof}
  Let $(X,f,t):\TorZ$. Since we are going to prove a proposition,
  we may take $t\jdeq\trunc e$ for some $e : (X,f)=(\zet,s)$.
  By induction in $e$, it suffices to prove that $P_p(\zet,s)$ is contractible.
  As center of contraction we take
  \[
    c_p(\zet,s,\trunc{\refl{(\zet,s)}})\defeq (a, h, q),
  \]
  where, for $x : \zet$ we set
  \[
    h(x) \defeq p^x : a = a,
  \]
  and $q$ is defined in a natural way by induction on $x : \zet$.

  It remains to prove that $(a,h,q)=(a',h',q')$ for all $(a',h',q')$,
  where $q'\,x : h'(f(x))=p\ct h'(x)$.
  Let $i\defeq h'(0): a=a'$.
  By transport lemmas we have $i_*\, h = (x\mapsto h(x)\ct i)$.
  We show $h(x) \ct i = h'(x)$ for all $x:X$, see \cref{fig:guided-null-hmtps},
  hence $i_*\, h = h'$.
  If $x \jdeq 0$, then this follows by $h(x)=p^x=p^0=\refl{a}$
  and the definition of $i$.
  For $x > 0$ we do induction on $x$,
  the base case having already been treated.
  Assume $h(x)\ct i = h'(x)$, and let $y = x+1$.
  Then $h(y) = h(x+1) = p \ct h(x)$,
  so $h(y)\ct i = p \ct h(x) \ct i = p\ct h'(x) = h'(f(x)) = h'(y)$.
  The case $x < 0$ is done by a similar induction.

  We also need to check that relative to this identification of $h$ and $h'$ over $i$
  we can identify $q$ and $q'$.
  We leave this for the reader, who may consult our formalization as needed.
\end{proof}

\begin{figure}
\begin{tikzpicture} %\large
   \matrix (m) 
   [matrix of math nodes, row sep=4em, column sep=3em, ampersand replacement=\&]
    { 
\ldots  \&  a' \&  \jdeq  \&  a' \& \jdeq \& a' \& \ldots\\
\ldots  \&  a  \&         \&  a  \&       \& a  \& \ldots\\
\ldots  \&  a  \&  \jdeq  \&  a  \& \jdeq \& a  \& \ldots\\};
\draw[->] (m-2-2) -- (m-1-2)  node[midway,right]  {$h'(f^{-1}(x))$};   
\draw[->] (m-2-4) -- (m-1-4)  node[midway,right]  {$h'(x)$}; 
\draw[->] (m-2-6) -- (m-1-6)  node[midway,right]  {$h'(f(x))$};     
\draw[->]  (m-2-4) -- (m-2-2)  node[midway,above] {$p$};   
\draw[->]  (m-2-6) -- (m-2-4)  node[midway,above] {$p$};
\draw[->] (m-2-2) -- (m-3-2)  node[midway,right]  {$h(f^{-1}(x))$};  
\draw[->] (m-2-4) -- (m-3-4)  node[midway,right]  {$h(x)$};
\draw[->] (m-2-6) -- (m-3-6)  node[midway,right]  {$h(f(x))$};   
\draw[->] (m-3-2) to[bend left=30] node[near end,left]{$i$} (m-1-2);
\draw[->] (m-3-4) to[bend left=30] node[near end,left]{$i$} (m-1-4);
\draw[->] (m-3-6) to[bend left=30] node[near end,left]{$i$} (m-1-6);                     
\end{tikzpicture}
\caption{\label{fig:guided-null-hmtps}Guided null homotopies}
\end{figure}

\begin{definition}\label{def:TBN}
For every $Z:\TorZ$, define
\begin{align*}\label{eq:TBD}
\hat{p}(Z)&\defeq (a,h), 
\text{~where $c_p(Z)\jdeq (a,h,!)$;}\\
\bar{p}(Z)&\defeq \pi_1 (\hat{p}(Z)).
\end{align*}
\end{definition}

Recall $\pt\defeq (\zet,s,\trunc{\refl{(\zet,s)}})$.
A straightforward calculation gives 
$\bar{p}(\pt)\jdeq \pi_1 (c_p(\pt)) \jdeq a$, which is one important
aspect of the universal property that we are after.
We also have $\hat{p}(\pt)\jdeq (a,n\mapsto p^n)$. CHECK
Before we can deal with the other aspect we have to prepare first.

There is a natural path $\pt =_\TorZ \pt$ obtained from $s$
by applying UA.
Using \cite[Theorem 2.7.2]{hottbook} we can analyse this path
componentwise. 

The first component must be a path of type $\zet=\zet$,
and we denote this path by $ua(s)$. 

The second component must now be
a path of type $ua(s)_*\, s = s$. The type family in which the transport
takes place is $X\mapsto (X\to X)$. This means transport by conjugation,
so we have to prove $s\circ s \circ s^{-1} = s$, which is easy.

The third component is a path in a contractible type, which always exists.

Instead of $(ua(s),!,!)$ we denote the path by $\Zloop$.
Modulo the equivalence from \cite[Theorem 2.7.2]{hottbook} we can wrap up:
\[
\Zloop : \pt =_\TorZ \pt
%\Zloop : (\zet,s,\trunc{\refl{(\zet,s)}}) = (\zet,s,\trunc{\refl{(\zet,s)}})
\]

\begin{lemma}\label{lem:TBN}
The following equations are well-typed and true:
\[
(a,n\mapsto p^{n-1}) = \Zloop_*\, (a,n\mapsto p^n) = (a,n\mapsto p^n).
\]
\end{lemma}
\begin{proof}
We have $\apd{\hat{p}}(\Zloop) : \Zloop_*\,\hat{p}(\pt) = \hat{p}(\pt)$,
and $\hat{p}(\pt) \jdeq (a,n\mapsto p^n)$. This proves the second equation.
The first equation follows by a careful analysis of the type families
in which the transport takes place. In constant type families, transport
is the identity function. In the type family $X\mapsto (X\to a=a)$,
the transport along $ua(s)$ amounts to precomposition with $s^{-1}$,
which proves the first equation.
\end{proof}

For the following lemma we once more 
invoke \cite[Theorem 2.7.2]{hottbook}, in particular
the equivalence $pair^=$ and its inverse $f$ from the proof.

\begin{lemma}\label{lem:TBN}
For all $q:(a,n\mapsto p^{n-1}) = (a,n\mapsto p^n)$ we have $q=pair^=(p,!)$.
\end{lemma}
\begin{proof}
The first component of $f(q)$ is a path $p':a=a$,
and the second component is a path 
$p'_*\, (n\mapsto p^{n-1}) = (a,n\mapsto p^n)$.
Again we must be careful about the type family in which the
transport takes place. This is the type
family $x\mapsto \zet\to a=x$. 
This means the transport along $p'$ is composition with $p'$ on the right.
Taking $n=0$ yields $p^{-1}\ct p' = \refl{a}$, so $p'=p$.
It follows that $q=pair^=(p,!)$.
\end{proof}

Combining several of the previous results with 
$\bar{p}\jdeq \pi_1 \circ \hat{p}$, and abstracting from the
parameters, we obtain a recursor for $\TorZ$.

\begin{definition}\label{def:TorZrecursor}
The function $e_A(a,p) \defeq \bar{p}$ has type
\[
e_A : \sum_{a:A}(a=a) \to (\TorZ \to A),
\]
plus $e_A(a,p)(\pt) \jdeq \bar{p}(\pt) \jdeq a$ and 
$\ap{e_A(a,p)}(\Zloop)\jdeq\ap{\bar{p}}(\Zloop) = p$.
\end{definition}

The above definition is one half of a universal property.
The other half is showing that $e_A$ is an equivalence.
The obvious candidate for an inverse is 
$g\mapsto (g(\pt),\ap{g}(\Zloop))$.
The homotopy for the roundtrip starting in $\sum_{a:A}(a=a)$
is precisely the justification of \cref{def:TorZrecursor}.
The other homotopy amounts to showing
$e_A(g(\pt),\ap{g}(\Zloop)) = g$, for any $g:\TorZ\to A$.
We know already that both functions agree in $\pt$,
and that $\TorZ$ is connected, so that all function values
are merely equal. This simple argument would suffice if
$A$ is a set but not in general. We are facing a similar
problem as was solved in the proof of \cref{lem:guided-null-hmtps}.
The idea is to go one dimension up with the null homotopies.
First some auxiliary notations.

\begin{definition}\label{def:loop-s-iterated}
For every $Z\defeq(X,f,t):\TorZ$ and $x:X$, 
define $s^Z_x: \pt =_\TorZ Z$ by $n\mapsto f^n(x)$ using UA. 
Using $t$, one can check $\isEq(n\mapsto f^n(x))$
and the other propositions involved. 
For simplicity, we denote $f^n(x)$ by $x+n$.
\end{definition}

The following lemma is proved by routine inductions.

\begin{lemma}\label{lem:loop-s-iterated}
For every $Z\defeq(X,f,t):\TorZ$, $x:X$ and $n:\zet$, 
we have $s^Z_{x+n} = \Zloop^n \ct s^Z_x$. 
In particular, we have a function $\varepsilon(n): s^\pt_n = \Zloop^n$.
\end{lemma}

The following lemma uses pathovers and their composition, 
which should be defined somewhere else. 

\begin{lemma}\label{lem:p_loop-iterated}
For every $P:\TorZ \to \UU$, $p_\pt : P(\pt)$, 
$p_\Zloop : \pt =^P_\Zloop \pt$ and $n:\zet$, 
we have $p_\Zloop^n :  \pt =^P_{\Zloop^n} \pt$.
\end{lemma}

Instead of the conclusion of the last lemma, we want to have
an inhabitant of $\pt =^P_{s^\pt_n} \pt$. This means we have
to correct for the error term $\varepsilon(n)$ in \cref{lem:loop-s-iterated}.
This can be done with \cref{lem:pathover-change-path}.

% Move elsewhere
\begin{lemma}\label{lem:pathover-change-path}
  For any $A : \UU$, $B : A \to \UU$, $a_1,a_2 : A$, $b_1 : B\,a_1$, $b_2: B\,a_2$,
  and paths $p,q:a_1=a_2$, and a $2$-dimensional path $\alpha : p = q$,
  we have an induced equivalence $(b_1 =^B_p b_2) \equiv (b_1 =^B_q b_2)$.
\end{lemma}
The proof is by induction on $\alpha$, after which we can take the identity equivalence.



\bibliographystyle{amsplain}
\bibliography{papers}
% \printindex
\end{document}
% Local Variables:
% fill-column: 144
% latex-block-names: ("lemma" "theorem" "remark" "definition" "corollary" "fact" "properties" "conjecture" "proof" "question" "proposition")
% TeX-master: t
% End:

\begin{lemma}\label{lem:redo-g-null-hmtps}
For all $a:A$, $p:a=a$, $g:\TorZ\to A$,
$q: g(\pt) = a$, $r: q_*(\ap{g}(\Zloop)) = p$, 
and $(X,f,t):\TorZ$, the type $P^1_p(X,f)$ is contractible.
\end{lemma}
\begin{proof}
TBD
\end{proof}

