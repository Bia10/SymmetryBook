%% packages
\usepackage[numbers]{natbib}
\usepackage{mathtools}          %to get \vcentcolon
\usepackage{letltxmacro}        %to rename \equiv
\LetLtxMacro{\oldequiv}{\equiv}
\usepackage{xspace}
\usepackage{tikz}

% hyperref should be the package loaded last
\usepackage[backref=page,
            colorlinks,
            citecolor=linkcolor,
            linkcolor=linkcolor,
            urlcolor=linkcolor,
            unicode,
            pdfauthor={CAS},
            pdftitle={Symmetry},
            pdfsubject={Mathematics},
            pdfkeywords={type theory, group theory, univalence axiom}]{hyperref}
% - except for cleveref!
\usepackage[capitalize]{cleveref}

\definecolor{linkcolor}{rgb}{0,0,0.5}

%% macros
\newcommand{\mytitle}{Symmetry}
\newcommand{\myauthor}{}

%%%%%%%%%%%%%%%%%%%%%%%%%%%%%%%%%%%%%%%%%%%%%%%%%%%%%%%%%%%%%%%%%%%%%%%%%%%%%
%% this part is from the AMS Bulletin template file, bull-l-template.tex
\newtheorem{theorem}{Theorem}  \numberwithin{theorem}{chapter}
 \newtheorem{lemma}[theorem]{Lemma}
 \newtheorem{axiom}[theorem]{Axiom}
\theoremstyle{definition}
    \newtheorem{definition}[theorem]{Definition}
    \newtheorem{example}[theorem]{Example}
    \newtheorem{xca}[theorem]{Exercise}
\theoremstyle{remark}
    \newtheorem{remark}[theorem]{Remark}
\numberwithin{equation}{section}
%% end
%%%%%%%%%%%%%%%%%%%%%%%%%%%%%%%%%%%%%%%%%%%%%%%%%%%%%%%%%%%%%%%%%%%%%%%%%%%%%

\newcommand{\arxiv}[1]{preprint available at \href{http://arxiv.org/abs/#1}{arXiv:#1}}

% Should these be roman or italic?!
\newcommand{\bool}{\mathop{\mathit{Bool}}}
\newcommand{\yes}{\mathop{\mathit{yes}}}
\newcommand{\no}{\mathop{\mathit{no}}}
\newcommand{\set}[1]{\{#1\}}
\newcommand{\refl}[1]{\mathop{{\it refl}_{#1}}}
\newcommand{\inl}[1]{\mathop{{\it inl}_{#1}}}
\newcommand{\inr}[1]{\mathop{{\it inr}_{#1}}}
\newcommand{\true}{\mathop{\mathit{True}}}
\newcommand{\false}{\mathop{\mathit{False}}}
\newcommand{\triv}{\mathop{\mathit{triv}}}
\newcommand{\symm}{\mathop{\mathit{symm}}}
\newcommand{\trans}{\mathop{\mathit{trans}}}
\newcommand{\trp}{\mathop{\mathit{transport}}}
\newcommand{\fst}{\mathop{\mathit{fst}}}
\newcommand{\snd}{\mathop{\mathit{snd}}}

\newcommand{\id}{\mathord{\mathrm{id}}}
\newcommand{\pt}{{\mathord{\mathrm{pt}}}}
\newcommand{\Type}{\mathord{\mathrm{Type}}}
\newcommand{\Set}{\mathord{\mathrm{Set}}}
\newcommand{\Group}{\mathord{\mathrm{Group}}}
\newcommand{\Gerbe}{\mathord{\mathrm{Gerbe}}}

\newcommand{\B}{\operatorname{B}\!} % use to combine with an operator and
                              % give a new operator with no extra space
\newcommand{\weq}{\simeq}
\newcommand{\QQ}{\mathbb{Q}}
\newcommand{\ZZ}{\mathbb{Z}}
\newcommand{\NN}{\mathbb{N}}
\newcommand{\CC}{\mathbb{C}}
\newcommand{\RR}{\mathbb{R}}
\newcommand{\isom}{\cong}
\newcommand*{\dblslash}{\mathbin{/\kern-3pt/}}


\renewcommand{\equiv}{\simeq}
\newcommand{\jdeq}{\oldequiv}
\newcommand{\defeq}{\vcentcolon\jdeq}

\DeclareMathOperator\Aut{Aut}
\DeclareMathOperator\Out{Out}
\DeclareMathOperator\Inn{Inn}
\DeclareMathOperator\Sym{Sym}
\DeclareMathOperator\Card{Card}
\DeclareMathOperator\gerbe{gerbe}
\DeclareMathOperator\fiber{fiber}

\DeclarePairedDelimiter\Trunc{\lVert}{\rVert}
\DeclarePairedDelimiter\trunc{\lvert}{\rvert} % truncation
\DeclarePairedDelimiter\merely{\lVert}{\rVert_{-1}}
\DeclarePairedDelimiter\angled{\langle}{\rangle}
\DeclarePairedDelimiter\Fin[]

\newcommand{\nonempty}[1]{\Trunc{#1}}

\newcommand\blfootnote[1]{%
  \begingroup
  \renewcommand\thefootnote{}\footnote{#1}%
  \addtocounter{footnote}{-1}%
  \endgroup}



%%%%%%%%%%%%%%%%%%%%%%%%%%%%%%%%%%%%%%%%%%%%%%%%%%%%%%%%%%%%%%%%%%%%%%%%%%%%%
% these arise from the group theory chapter
%%%%%%%%%%%%%%%%%%%%%%%%%%%%%%%%%%%%%%%%%%%%%%%%%%%%%%%%%%%%%%%%%%%%%%%%%%%%%

%%%%%%%%%%%%%%%%%%%%%%%%%%%%%%%%%%%%%%%%%%%%%%%%%%%%%%%%%%%%%%%%%%%%%%%%%%%%
%originates in group.tex (BID)
\newcommand{\ie}{{\em i.e., }}%\xspace}fixlater
\newcommand{\defequi}{\defeq}%definitionally equal}
\newcommand{\iscontr}{\mathrm{isContr}}
\newcommand{\isEq}{\mathrm{isEquiv}}
\newcommand{\isset}{\mathrm{isSet}}
\newcommand{\isonetype}{\mathrm{1Type}}
\newcommand{\isconn}{\mathrm{isConn}}
\newcommand{\conn}{\mathrm{conn}}
\newcommand{\aut}{\mathrm{Aut}}
\newcommand{\Hom}{\mathrm{Hom}}
\newcommand{\setgroup}[1]{||#1||}
\newcommand{\inftygp}{$\infty$-group\xspace}
\newcommand{\aninftygp}{an $\infty$-group\xspace}
\newcommand{\inftygps}{$\infty$-groups\xspace}
\newcommand{\princ}{\mathrm{Princ}}
%some special types
\newcommand{\Gtorsor}{\mathrm{tors}_G}
\newcommand{\typegroup}{\mathbf{Group}}
\newcommand{\typeset}{\mathbf{Set}}
\newcommand{\typeinftygp}{{\infty}\mathbf{Group}}
\newcommand{\pttype}{\mathcal U_*}
\newcommand{\typeabsgp}{\mathbf{Group}_{\mathrm{Abstract}}}
\newcommand{\typemonoid}{\mathbf{Monoid}}
\newcommand{\typetorsor}{\mathbf{Tors}}



%%%%%%%%%%%%%%%%%%%%%%%%%%%%%%%%%%%%%%%%%%%%%%%%%%%%%%%%%%%%%%%%%%%%%%%%%%%%

%%% Local Variables:
%%% mode: latex
%%% TeX-master: "book"
%%% End:
