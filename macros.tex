%% macros
\usepackage[numbers]{natbib}
\newcommand{\mytitle}{Symmetry}
\newcommand{\myauthor}{}

%%%%%%%%%%%%%%%%%%%%%%%%%%%%%%%%%%%%%%%%%%%%%%%%%%%%%%%%%%%%%%%%%%%%%%%%%%%%%
%% this part is from the AMS Bulletin template file, bull-l-template.tex
\newtheorem{theorem}{Theorem}  \numberwithin{theorem}{chapter}
 \newtheorem{lemma}[theorem]{Lemma}
 \newtheorem{axiom}[theorem]{Axiom}
\theoremstyle{definition}
    \newtheorem{definition}[theorem]{Definition}
    \newtheorem{example}[theorem]{Example}
    \newtheorem{xca}[theorem]{Exercise}
\theoremstyle{remark}
    \newtheorem{remark}[theorem]{Remark}
\numberwithin{equation}{section}
%% end
%%%%%%%%%%%%%%%%%%%%%%%%%%%%%%%%%%%%%%%%%%%%%%%%%%%%%%%%%%%%%%%%%%%%%%%%%%%%%

\newcommand{\arxiv}[1]{preprint available at \href{http://arxiv.org/abs/#1}{arXiv:#1}}

% Should these be roman or italic?!
\newcommand{\bool}{\mathop{\mathit{Bool}}}
\newcommand{\yes}{\mathop{\mathit{yes}}}
\newcommand{\no}{\mathop{\mathit{no}}}
\newcommand{\refl}[1]{\mathop{{\it refl}_{#1}}}
\newcommand{\true}{\mathop{\mathit{True}}}
\newcommand{\false}{\mathop{\mathit{False}}}
\newcommand{\triv}{\mathop{\mathit{triv}}}
\newcommand{\symm}{\mathop{\mathit{symm}}}
\newcommand{\trans}{\mathop{\mathit{trans}}}

\newcommand{\pt}{\mathord{\mathrm{pt}}}
\newcommand{\Type}{\mathord{\mathrm{Type}}}

\newcommand{\weq}{\simeq}
\newcommand{\QQ}{\mathbb{Q}}
\newcommand{\ZZ}{\mathbb{Z}}
\newcommand{\NN}{\mathbb{N}}
\newcommand{\CC}{\mathbb{C}}
\newcommand{\RR}{\mathbb{R}}
\newcommand{\isom}{\cong}
\newcommand*{\dblslash}{\mathbin{/\kern-3pt/}}


\usepackage{mathtools}          %to get \vcentcolon
\usepackage{letltxmacro}        %to rename \equiv
\LetLtxMacro{\oldequiv}{\equiv}
\renewcommand{\equiv}{\simeq}
\newcommand{\jdeq}{\oldequiv}
\newcommand{\defeq}{\vcentcolon\jdeq}

\DeclareMathOperator\Aut{Aut}
\DeclareMathOperator\Sym{Sym}
\DeclareMathOperator\Card{Card}

\DeclarePairedDelimiter\Trunc{\lVert}{\rVert}
\DeclarePairedDelimiter\trunc{\lVert}{\rVert} % truncation
\DeclarePairedDelimiter\merely{\lVert}{\rVert_{-1}}
\DeclarePairedDelimiter\angled{\langle}{\rangle}
\DeclarePairedDelimiter\Fin[]

\newcommand{\nonempty}[1]{\Trunc{#1}}

\newcommand\blfootnote[1]{%
  \begingroup
  \renewcommand\thefootnote{}\footnote{#1}%
  \addtocounter{footnote}{-1}%
  \endgroup}


\usepackage{hyperref} % should be the package loaded last
\hypersetup{pdftitle={\mytitle}}
\hypersetup{pdfauthor={\myauthor}}

%%%%%%%%%%%%%%%%%%%%%%%%%%%%%%%%%%%%%%%%%%%%%%%%%%%%%%%%%%%%%%%%%%%%%%%%%%%%%
% these arise from the group theory chapter
%%%%%%%%%%%%%%%%%%%%%%%%%%%%%%%%%%%%%%%%%%%%%%%%%%%%%%%%%%%%%%%%%%%%%%%%%%%%%

%%% Local Variables:
%%% mode: latex
%%% TeX-master: "book"
%%% End:
