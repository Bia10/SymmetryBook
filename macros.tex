%% packages
\usepackage[numbers]{natbib}
\usepackage{mathtools}          %to get \vcentcolon
\usepackage{letltxmacro}        %to rename \equiv
\LetLtxMacro{\oldequiv}{\equiv}
\usepackage{xspace}
\usepackage[all]{xy}
\usepackage{tikz}
\usetikzlibrary{decorations.pathreplacing,matrix,arrows,chains,positioning,scopes} 
        %,hobby} % for hobby splines
%% useful for debugging bezier paths
%\tikzset{%
%  show curve controls/.style={
%    postaction={
%      decoration={
%        show path construction,
%        curveto code={
%          \draw [blue] 
%            (\tikzinputsegmentfirst) -- (\tikzinputsegmentsupporta)
%            (\tikzinputsegmentlast) -- (\tikzinputsegmentsupportb);
%          \fill [red, opacity=0.5] 
%            (\tikzinputsegmentsupporta) circle [radius=.5ex]
%            (\tikzinputsegmentsupportb) circle [radius=.5ex];
%        }
%      },
%      decorate
%    }}}


% hyperref should be the package loaded last
\usepackage[backref=page,
            colorlinks,
            citecolor=linkcolor,
            linkcolor=linkcolor,
            urlcolor=linkcolor,
            unicode,
            pdfauthor={CAS},
            pdftitle={Symmetry},
            pdfsubject={Mathematics},
            pdfkeywords={type theory, group theory, univalence axiom}]{hyperref}
% - except for cleveref!
\usepackage[capitalize]{cleveref}

\definecolor{linkcolor}{rgb}{0,0,0.5}

%% macros
\newcommand{\mytitle}{Symmetry}
\newcommand{\myauthor}{}

%%%%%%%%%%%%%%%%%%%%%%%%%%%%%%%%%%%%%%%%%%%%%%%%%%%%%%%%%%%%%%%%%%%%%%%%%%%%%
%% this part is from the AMS Bulletin template file, bull-l-template.tex
\newtheorem{theorem}{Theorem}  \numberwithin{theorem}{chapter}
 \newtheorem{lemma}[theorem]{Lemma}
 \newtheorem{axiom}[theorem]{Axiom}
 \newtheorem{corollary}[theorem]{Corollary}
\theoremstyle{definition}
    \newtheorem{definition}[theorem]{Definition}
    \newtheorem{example}[theorem]{Example}
    \newtheorem{xca}[theorem]{Exercise}
\theoremstyle{remark}
    \newtheorem{remark}[theorem]{Remark}
\numberwithin{equation}{section}
%% end
%%%%%%%%%%%%%%%%%%%%%%%%%%%%%%%%%%%%%%%%%%%%%%%%%%%%%%%%%%%%%%%%%%%%%%%%%%%%%

\newcommand{\arxiv}[1]{preprint available at \href{http://arxiv.org/abs/#1}{arXiv:#1}}

% Should these be roman or italic?!
\newcommand{\bool}{\mathop{\mathit{Bool}}}
\newcommand{\yes}{\mathop{\mathit{yes}}}
\newcommand{\no}{\mathop{\mathit{no}}}
\newcommand{\set}[1]{\{#1\}}
\newcommand{\refl}[1]{\mathop{{\it refl}_{#1}}}
\newcommand{\inl}[1]{\mathop{{\it inl}_{#1}}}
\newcommand{\inr}[1]{\mathop{{\it inr}_{#1}}}
\newcommand{\true}{\mathop{\mathit{True}}}
\newcommand{\false}{\mathop{\mathit{False}}}
\newcommand{\triv}{\mathop{\mathit{triv}}}
\newcommand{\symm}{\mathop{\mathit{symm}}}
\newcommand{\trans}{\mathop{\mathit{trans}}}
\newcommand{\fact}{\mathop{\mathit{fact}}}
\newcommand{\trp}{\mathord{\mathit{transport}}}
\newcommand{\fst}{\mathop{\mathit{fst}}}
\newcommand{\snd}{\mathop{\mathit{snd}}}
\newcommand{\zpos}{\mathop{\mathrm{pos}}}
\newcommand{\zneg}{\mathop{\mathrm{neg}}}
\newcommand{\zzero}{\mathop{\mathrm{zero}}}
\newcommand{\ap}[1]{\mathop{{\it ap}_{#1}}}
\newcommand{\apd}[1]{\mathop{{\it apd}_{#1}}}

\newcommand{\id}{\mathord{\mathrm{id}}}
\newcommand{\pt}{{\mathord{\mathrm{pt}}}}
\newcommand{\Type}{\mathord{\mathrm{Type}}}
\newcommand{\Set}{\mathord{\mathrm{Set}}}
\newcommand{\Group}{\mathord{\mathrm{Group}}}
\newcommand{\AbGroup}{\mathord{\mathrm{AbGroup}}}
\newcommand{\Gerbe}{\mathord{\mathrm{Gerbe}}}
\newcommand{\AbGerbe}{\mathord{\mathrm{AbGerbe}}}
\newcommand{\Band}{\mathord{\mathrm{Band}}}
\newcommand{\AbBand}{\mathord{\mathrm{AbBand}}}

\newcommand{\B}{\operatorname{B}\!} % use to combine with an operator and
                              % give a new operator with no extra space
\newcommand{\weq}{\simeq}
\newcommand{\QQ}{\mathbb{Q}}
\newcommand{\ZZ}{\mathbb{Z}}
\newcommand{\NN}{\mathbb{N}}
\newcommand{\CC}{\mathbb{C}}
\newcommand{\RR}{\mathbb{R}}
\newcommand{\isom}{\cong}
\newcommand{\ct}{*}
\newcommand{\cto}{*_{\mathrm{o}}}
\newcommand*{\dblslash}{\mathbin{/\kern-3pt/}}

\renewcommand{\equiv}{\simeq}
\newcommand{\liff}{\equiv}
\newcommand{\jdeq}{\oldequiv}
\newcommand{\defeq}{\vcentcolon\jdeq}
\newcommand{\defequi}{\defeq}%definitionally equal}

\DeclareMathOperator\Aut{Aut}
\DeclareMathOperator\Out{Out}
\DeclareMathOperator\Inn{Inn}
\DeclareMathOperator\Ker{Aut}
\DeclareMathOperator\Sym{Sym}
\DeclareMathOperator\Card{Card}
\DeclareMathOperator\gerbe{gerbe}
\DeclareMathOperator\band{band}
\DeclareMathOperator\fiber{fiber}
\DeclareMathOperator\Succ{Succ}
\DeclareMathOperator\fin{Fin}

\DeclarePairedDelimiter\Trunc{\lVert}{\rVert}
\DeclarePairedDelimiter\trunc{\lvert}{\rvert} % truncation
\DeclarePairedDelimiter\merely{\lVert}{\rVert_{-1}}
\DeclarePairedDelimiter\angled{\langle}{\rangle}
\DeclarePairedDelimiter\Fin[]
\DeclarePairedDelimiterX\setof[2]\lbrace\rbrace{#1 \mid #2}

\newcommand{\nonempty}[1]{\Trunc{#1}}

\newcommand\blfootnote[1]{%
  \begingroup
  \renewcommand\thefootnote{}\footnote{#1}%
  \addtocounter{footnote}{-1}%
  \endgroup}



%%%%%%%%%%%%%%%%%%%%%%%%%%%%%%%%%%%%%%%%%%%%%%%%%%%%%%%%%%%%%%%%%%%%%%%%%%%%%
% these arise from the group theory chapter
%%%%%%%%%%%%%%%%%%%%%%%%%%%%%%%%%%%%%%%%%%%%%%%%%%%%%%%%%%%%%%%%%%%%%%%%%%%%%

%%%%%%%%%%%%%%%%%%%%%%%%%%%%%%%%%%%%%%%%%%%%%%%%%%%%%%%%%%%%%%%%%%%%%%%%%%%%
%originates in group.tex (BID)
\newcommand{\ie}{{\em i.e., }}%\xspace}fixlater
\newcommand{\ev}{\mathrm{ev}}


\newcommand{\iscontr}{\mathrm{isContr}}
\newcommand{\isprop}{\mathrm{isProp}}
\newcommand{\isset}{\mathrm{isSet}}
\newcommand{\isgrpd}{\mathrm{isGrpd}}
\newcommand{\isEq}{\mathrm{isEquiv}}
\newcommand{\isonetype}{\mathrm{1Type}}
\newcommand{\isconn}{\mathrm{isConn}}

\newcommand{\conn}{\mathrm{conn}}
\newcommand{\aut}{\mathrm{Aut}}
\newcommand{\Hom}{\mathrm{Hom}}
\newcommand{\setgroup}[1]{||#1||}
\newcommand{\inftygp}{$\infty$-group\xspace}
\newcommand{\aninftygp}{an $\infty$-group\xspace}
\newcommand{\inftygps}{$\infty$-groups\xspace}
\newcommand{\princ}{\mathrm{Princ}}
%some special types
\newcommand{\Gtorsor}{\mathrm{tors}_G}
\newcommand{\Xtorsor}[1]{\mathrm{tors}_{#1}}%      added by MAB
\newcommand{\Ztorsor}{\Xtorsor{\zet}}
\newcommand{\TorZ}{{\mathrm{T}\ZZ}} % {\mathbf{TorZor}}% alternative: BZ.   end MAB
\newcommand{\typegroup}{\mathbf{Group}}
\newcommand{\typeset}{\Set}
\newcommand{\Prop}{\mathrm{Prop}}
\newcommand{\typeinftygp}{{\infty}\mathbf{Group}}
\newcommand{\UU}{\mathcal{U}}
\newcommand{\pttype}{\UU_*}
\newcommand{\typeabsgp}{\mathbf{Group}_{\mathrm{Abstract}}}
\newcommand{\typemonoid}{\mathbf{Monoid}}
\newcommand{\typetorsor}{\mathbf{Tors}}
\newcommand{\BSigma}{B\Sigma}%previously \Set_{(S)} - the component of S:\Set
\newcommand{\twist}{\mathit{twist}}%loop in BC_2
\newcommand{\base}{\bullet}%point in circle
\newcommand{\Zloop}{\mathrm{loop}}%MAB: loop TorZor
\newcommand{\Sloop}{\mathit{loop}}%loop in circle
\newcommand{\bn}[1]{\mathbf{#1}}
\newcommand{\Eq}{\mathrm{Eq}}

\newcommand{\idtoeq}{\mathrm{id2eq}}%this is a provocation: fix it if you prefer
\newcommand{\ua}{\mathrm{ua}}%univalence inverse
\newcommand{\zet}{\mathrm{Zet}}%the SET of integers  this is a provocation: fix it if you prefer

%%%%%%%%%%%%%%%%%%%%%%%%%%%%%%%%%%%%%%%%%%%%%%%%%%%%%%%%%%%%%%%%%%%%%%%%%%%%

%%% Local Variables:
%%% mode: latex
%%% TeX-master: "book"
%%% End:
