\section{Cayley diagram}

$G \equiv \Aut(D_G) \to \Sym(\Card G)$

\section{Actions}

If $G$ is any (possibly higher) group and $A$ is any type of objects,
then we define an \emph{action} by $G$ on an $A$ as a function
\[
  X : BG \to A.
\]
The particular $A$ being acted on is $X(\pt):A$,
and the action itself is given by transport.

Many times we are particularly interested in actions on types,
i.e., $A$ is a universe (or the universe of types-at-large):
\[
  X : BG \to \Type.
\]

In this case, we define \emph{orbit type} of the action as
\[
  X_G \defeq \Sigma_{z:BG} X(z),
\]
and the type of \emph{fixed points} as
\[
  X^G \defeq \Pi_{z:BG} X(z).
\]
The set of orbits is the set-truncation of the orbit type,
\[
  X / G \defeq \Trunc{X_G}_0.
\]

\section{Orbit-stabilizer theorem}

Given an action $X : BG \to \Type$ and a point $x : X(\pt)$, we define
the \emph{orbit} through $x$ as the subtype of $X(\pt)$ consisting of
all $y : X(\pt)$ that are merely equal to $x$ in the orbit type:
\[
  G\cdot x \defeq \mathcal O_x \defeq \Sigma_{y : X(\pt)} \merely{[x] = [y]}
\]
(Note the unfortunate terminology: an orbit is not an element in the
orbit type!)
Note that this only depends on the image of $x$ in the set of orbits,
thus justifying the names.

In this way, the type $X(\pt)$ splits as a disjoint union of orbits,
indexed by the set of orbits
\[
  X(\pt) \equiv \sqcup_{z : X / G} \mathcal O_z.
\]

The \emph{stabilizer group} $G_x$ of $x : X(\pt)$ is the automorphism group of $[x]$ in the orbit type.
Different points in the same orbit have conjugate stabilizer groups.

\begin{theorem}[Orbit-stabilizer theorem]
  There is a canonical action $\tilde G : BG_x \to \Type$,
  acting on $\tilde G(\pt)\equiv G$
  with orbit type $\tilde G\dblslash G_x \simeq \mathcal O_x$.
\end{theorem}
\begin{proof}
  Define $\tilde G(x,y,!) \defeq (\pt = x)$.
\end{proof}

Now suppose that $G$ is a $1$-group acting on a set.
We see that the orbit type is a set
(and is thus equivalent to the set of orbits)
if and only if
all stabilizer groups are trivial.

If $G$ is a $1$-group,
then so is each stabilizer-group,
and in this case (of a set-action),
the orbit-stabilizer theorem
tells us that 

\begin{theorem}[Lagrange's Theorem]
  If $H \to G$ is a subgroup, then $H$ has a natural action on $G$,
  and all the orbits under this action are equivalent.
\end{theorem}

\section{(the lemma that is not) Burnside's lemma}

%%% Local Variables:
%%% mode: latex
%%% fill-column: 144
%%% TeX-master: "book"
%%% End:
