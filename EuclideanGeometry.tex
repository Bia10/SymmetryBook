%% \section{scalar products}
%% \section{euclidean frames, relation to determinants(?)}
%% \section{the euclidean group as a semidirect product}
%% \section{euclidean properties (length, angle, etc.)}


In this chapter we study Euclidean geometry.  We assume some standard linear
algebra over real numbers, including the notion of finite dimensional vector
space over the real numbers and the notion of inner product.

\section{Euclidean spaces}

\begin{definition}\label{def:InnerProductSpace}
  An {\em inner product space} $V$ is a real vector space of finite dimension
  equipped with an inner product $H : (x,y) \mapsto H( x,y )$.
\end{definition}

Let $\OS$ denote the type of inner product spaces.

For each natural number $n$, we may construct the {\em standard} inner product
space $\VV^n \defeq (V,H)$ of dimension $n$ as follows.  For $V$ we take the
vector space $\RR^n$, and we equip it with the standard inner product given by
the dot product
$$ H ( x , y) \defeq x \cdot y, $$
where the dot product is defined as usual as
$$ x \cdot y \defeq \sum_i x_i y_i . $$

\begin{theorem}\label{thm:GramSchmidt}
  Any inner product space $V$ is merely equal to $\VV^n$, where $n$ is $\dim E$.
\end{theorem}

\begin{proof}
  Since any finite dimensional vector space merely has a basis, we may assume
  we have a basis for $V$.  Now use Gram-Schmidt orthonormalization to show
  that $V = \VV^n$.
\end{proof}

\begin{lemma}\label{lem:InnerProductSpace1Type}
  The type $\OS$ is a $1$-type.
\end{lemma}

\begin{proof}
  Given two inner product spaces $V$ and $V'$, we must show that the type
  $V=V'$ is a set.  By univalence, its elements correspond to linear
  isomorphisms $f : V \xrightarrow \simeq V'$ that are compatible with the
  inner products.  Those form a set.
\end{proof}

\begin{definition}\label{def:OrthogonalGroup}
  Given a natural number $n$, we define the {\em orthogonal group} $\OrthGp n$
  by letting $\B (\OrthGp n)$ be the connected component of $\OS$ containing
  the point $\VV^n$ and equipping it with the proof that it is a connected
  groupoid provided by \cref{thm:GramSchmidt} and
  \cref{lem:InnerProductSpace1Type}.
\end{definition}

From \cref{thm:GramSchmidt}, we see that $\OS$ is equivalent to 
$\sum_n \B (\OrthGp n)$.

\begin{definition}\label{def:EuclideanSpace}
  A {\em Euclidean space} $E$ is a torsor $\Points E$ for the additive group
  underlying an inner product space $\Vectors E$.
\end{definition}

For each natural number $n$, we may construct the {\em standard} Euclidean
space $\EE^n$ of dimension $n$ as follows.  For $\Vectors E$ we take the
standard inner product space $\VV^n$, and for $\Points E$ we take the
corresponding principal torsor $\princ {\RR^n}$.

We denote the type of all Euclidean spaces by $\ES$.

For $E:\ES$, we let $P:E$ serve as notation for $P:\Points E$.  These will be
the {\em points} in the geometry of $E$.

\begin{theorem}\label{thm:EuclideanNormalization}
  Any Euclidean space $E$ is merely equal to $\EE^n$, where $n$ is $\dim E$.
\end{theorem}

\begin{proof}
  Since we are proving a proposition and any torsor is merely trivial, by
  \cref{thm:GramSchmidt} we may assume $\Vectors E$ is $\VV^n$.  Similarly, we
  may assume that $\Points E$ is the trivial torsor.
\end{proof}

\begin{lemma}\label{lem:EuclideanSpace1Type}
  The type $\ES$ is a $1$-type.
\end{lemma}

\begin{proof}
  Observe using \cref{lem:BGbytorsor} that $\ES \simeq \sum_n \sum_{V:\B (
    \OrthGp n )} \B V$, and that the types $\B ( \OrthGp n )$ and $\B V$ are
  1-types.
\end{proof}

\begin{definition}\label{def:EuclideanGroup}
  Given a natural number $n$, we define the {\em Euclidean group} $\EucGp n$ by
  letting $\B (\EucGp n)$ be the connected component of $\ES$ containing the
  point $\EE^n$ and equipping it with the proof that it is a connected
  groupoid provided by \cref{thm:EuclideanNormalization} and
  \cref{lem:EuclideanSpace1Type}.
\end{definition}

We see that $\ES$ is equivalent to the sum $\sum_n \B (\EucGp n)$.

\begin{theorem}\label{thm:EuclideanGroupSemidirect}
  For each $n$, the Euclidean group $\EucGp n$ is a semidirect product
  $\OrthGp n \ltimes \RR^n$.
\end{theorem}

\begin{proof}
  Recall \cref{def:semidirect-product} and observe that
  $$\EucGp n \simeq \sum_{V:\B ( \OrthGp n )} \B V,$$
  again using \cref{lem:BGbytorsor}.
\end{proof}
