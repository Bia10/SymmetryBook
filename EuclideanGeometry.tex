%% \section{scalar products}
%% \section{euclidean frames, relation to determinants(?)}
%% \section{the euclidean group as a semidirect product}
%% \section{euclidean properties (length, angle, etc.)}


In this chapter we study Euclidean geometry.  We assume some standard linear
algebra over real numbers, including the notion of finite dimensional vector
space over the real numbers and the notion of inner product.

\section{Inner product spaces}

\begin{definition}\label{def:InnerProductSpace}
  An {\em inner product space} $V$ is a real vector space of finite dimension
  equipped with an inner product $H : (x,y) \mapsto H( x,y )$.
\end{definition}

Let $\OS$ denote the type of inner product spaces.  It is a type of pairs whose
elements are of the form $(V,H)$.

For each natural number $n$, we may construct the {\em standard} inner product
space $\VV^n \defeq (V,H)$ of dimension $n$ as follows.  For $V$ we take the
vector space $\RR^n$, and we equip it with the standard inner product given by
the dot product
$$ H ( x , y) \defeq x \cdot y, $$
where the dot product is defined as usual as
$$ x \cdot y \defeq \sum_i x_i y_i . $$

\begin{theorem}\label{thm:GramSchmidt}
  Any inner product space $V$ is merely equal to $\VV^n$, where $n$ is $\dim E$.
\end{theorem}

\begin{proof}
  Since any finite dimensional vector space merely has a basis, we may assume
  we have a basis for $V$.  Now use Gram-Schmidt orthonormalization to show
  that $V = \VV^n$.
\end{proof}

\begin{lemma}\label{lem:InnerProductSpace1Type}
  The type $\OS$ is a $1$-type.
\end{lemma}

\begin{proof}
  Given two inner product spaces $V$ and $V'$, we must show that the type
  $V=V'$ is a set.  By univalence, its elements correspond to the linear
  isomorphisms $f : V \xrightarrow \weq V'$ that are compatible with the
  inner products.  Those form a set.
\end{proof}

\begin{definition}\label{def:OrthogonalGroup}
  Given a natural number $n$, we define the {\em orthogonal group} $\OrthGp n$
  by letting $\B (\OrthGp n)$ be the connected component of $\OS$ containing
  the point $\VV^n$ and equipping it with the proof that it is a connected
  groupoid provided by \cref{thm:GramSchmidt} and
  \cref{lem:InnerProductSpace1Type}.
\end{definition}

We may also omit the parentheses and write $\B\OrthGp n$.

From \cref{thm:GramSchmidt}, we see that $\OS$ is equivalent to 
$\sum_n \B \OrthGp n$.

\section{Euclidean spaces}

\begin{definition}\label{def:EuclideanSpace}
  A {\em Euclidean space} $E$ is a torsor $A$ for the additive group
  underlying an inner product space $V$.
\end{definition}

We will write $V$ also for the additive group underlying $V$.  Thus an
expression such as $\B V$ will denote the classifying space of the underlying
additive group\footnote{We are careful not to refer to the group as an Abelian
  group at this point, even though it is one, because the notation $\B$ may be
  used in some contexts to denote a different construction on Abelian groups.}
of $V$.

We denote the type of all Euclidean spaces by $\ES$.  It is a type of pairs
$E = (V,A) : \sum_{V:\OS} \typetorsor_V$.  We introduce the notation
$\Vectors E \defeq V$ and $\Points E \defeq A$.

For $E:\ES$, we let $P:E$ serve as notation for $P:\Points E$.  These will be
the {\em points} in the geometry of $E$.

We will think of the torsor in abstract terms: it is a nonempty set upon which
$V$ acts.  Since $V$ is an additive group, we prefer to write the action
additively, too: given $v:V$ and $P:E$ we have $v+P:E$.  Moreover, given
$P,Q:E$, there is a unique $v:V$ such $v+P = Q$; for it we introduce the
notation $Q-P \defeq v$.

For each natural number $n$, we may construct the {\em standard} Euclidean
space $\EE^n : \ES$ of dimension $n$ as follows.  For $\Vectors E$ we take the
standard inner product space $\VV^n$, and for $\Points E$ we take the
corresponding principal torsor $\princ {\RR^n}$.

\begin{theorem}\label{thm:EuclideanNormalization}
  Any Euclidean space $E$ is merely equal to $\EE^n$, where $n$ is $\dim E$.
\end{theorem}

\begin{proof}
  Since we are proving a proposition and any torsor is merely trivial, by
  \cref{thm:GramSchmidt} we may assume $\Vectors E$ is $\VV^n$.  Similarly, we
  may assume that $\Points E$ is the trivial torsor.
\end{proof}

\begin{lemma}\label{lem:EuclideanSpace1Type}
  The type $\ES$ is a $1$-type.
\end{lemma}

\begin{proof}
  Observe using \cref{lem:BGbytorsor} that
  $\ES \weq \sum_n \sum_{V:\B \OrthGp n} \B V$ and that the
  types $\B \OrthGp n$ and $\B V$ are 1-types.
\end{proof}

\begin{definition}\label{def:EuclideanGroup}
  Given a natural number $n$, we define the {\em Euclidean group} $\EucGp n$ by
  letting $\B \EucGp n$ be the connected component of $\ES$ containing the
  point $\EE^n$ and equipping it with the proof that it is a connected
  groupoid provided by \cref{thm:EuclideanNormalization} and
  \cref{lem:EuclideanSpace1Type}.
\end{definition}

We see that $\ES$ is equivalent to the sum $\sum_n \B \EucGp n$.

\begin{theorem}\label{thm:EuclideanGroupSemidirect}
  For each $n$, the Euclidean group $\EucGp n$ is equivalent to a semidirect
  product $\OrthGp n \ltimes \RR^n$.
\end{theorem}

\begin{proof}
  Recall \cref{def:semidirect-product} and apply it to the function $\tilde H :
  \B \OrthGp n \to \typegroup$ that sends an inner product space $V$ of
  dimension $n$ to its underlying additive group, also written, according to
  our convention, as $V$.  Since the value $\tilde H ( \VV^n )$ at the
  basepoint is $\RR^n$, the map $\tilde H$ is an action of the group
  $\OrthGp n$ on $\RR^n$, and the semidirect product $\OrthGp n \ltimes \RR^n$ has
  $\sum_{V:\B \OrthGp n} \B V$ as its underlying pointed type.
  Finally, observe that $\EucGp n \weq \sum_{V:\B \OrthGp n} \B V$, again
  using \cref{lem:BGbytorsor}.
\end{proof}
